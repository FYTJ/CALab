% 这是中国科学院大学计算机科学与技术专业《计算机组成原理(研讨课)》使用的实验报告 Latex 模板
% 本模板与 2024 年 2 月 Jun-xiong Ji 完成, 更改自由 Shing-Ho Lin 和 Jun-Xiong Ji 于 2022 年 9 月共同完成的基础物理实验模板
% 如有任何问题, 请联系: jijunxoing21@mails.ucas.ac.cn
% This is the LaTeX template for report of Experiment of Computer Organization and Design courses, based on its provided Word template.
% This template is completed on Febrary 2024, based on the joint collabration of Shing-Ho Lin and Junxiong Ji in September 2022.
% Adding numerous pictures and equations leads to unsatisfying experience in Word. Therefore LaTeX is better.
% Feel free to contact me via: jijunxoing21@mails.ucas.ac.cn

\documentclass[11pt]{article}

\usepackage[a4paper]{geometry}
\geometry{left=2.0cm,right=2.0cm,top=2.5cm,bottom=2.5cm}

\usepackage{ctex} % 支持中文的LaTeX宏包
\usepackage{amsmath,amsfonts,graphicx,subfigure,amssymb,bm,amsthm,mathrsfs,mathtools,breqn} % 数学公式和符号的宏包集合
\usepackage{algorithm,algorithmicx} % 算法和伪代码
\usepackage[noend]{algpseudocode} % 算法和伪代码
\usepackage{fancyhdr} % 自定义页眉页脚
\usepackage[framemethod=TikZ]{mdframed} % 创建带边框的框架
\usepackage{fontspec} % 字体设置
\usepackage{adjustbox} % 调整盒子大小
\usepackage{fontsize} % 设置字体大小
\usepackage{tikz,xcolor} % 绘制图形和使用颜色
\usepackage{multicol} % 多栏排版
\usepackage{multirow} % 表格中合并单元格
\usepackage{pdfpages} % 插入PDF文件
\usepackage{listings} % 在文档中插入源代码
\usepackage{wrapfig} % 文字绕排图片
\usepackage{bigstrut,multirow,rotating} % 支持在表格中使用特殊命令
\usepackage{booktabs} % 创建美观的表格
\usepackage{circuitikz} % 绘制电路图
\usepackage{zhnumber} % 中文序号(用于标题)
\usepackage{tabularx} % 表格折行

\definecolor{dkgreen}{rgb}{0,0.6,0}
\definecolor{gray}{rgb}{0.5,0.5,0.5}
\definecolor{mauve}{rgb}{0.58,0,0.82}
% \lstset{
%   frame=tb,
%   aboveskip=3mm,
%   belowskip=3mm,
%   showstringspaces=false,
%   columns=flexible,
%   framerule=1pt,
%   rulecolor=\color{gray!35},
%   backgroundcolor=\color{gray!5},
%   basicstyle={\small\ttfamily},
%   numbers=none,
%   numberstyle=\tiny\color{gray},
%   keywordstyle=\color{blue},
%   commentstyle=\color{dkgreen},
%   stringstyle=\color{mauve},
%   breaklines=true,
%   breakatwhitespace=true,
%   tabsize=3,
% }



\lstset{
    % basicstyle=\ttfamily\small,
    basicstyle=\ttfamily\footnotesize,
    keywordstyle=\color{blue},
    commentstyle=\color{gray},
    numbers=left,
    numberstyle=\tiny,
    breaklines=true,
    frame=single,
}

\lstdefinelanguage{Verilog}{
    morekeywords={always, and, assign, attribute, begin, buf, bufif0, bufif1, case, casex, casez,
                 cmos, deassign, default, defparam, disable, edge, else, end, endattribute,
                 endcase, endfunction, endmodule, endprimitive, endspecify, endtable, endtask,
                 event, for, force, forever, fork, function, highz0, highz1, if, ifnone, initial,
                 inout, input, integer, join, large, macromodule, medium, module, nand, negedge,
                 nmos, nor, not, notif0, notif1, or, output, parameter, pmos, posedge, primitive,
                 pull0, pull1, pulldown, pullup, rcmos, real, realtime, reg, release, repeat,
                 rnmos, rpmos, rtran, rtranif0, rtranif1, scalared, signed, small, specify,
                 specparam, strong0, strong1, supply0, supply1, table, task, time, tran, tranif0,
                 tranif1, tri, tri0, tri1, triand, trior, trireg, unsigned, vectored, wait, wand,
                 weak0, weak1, while, wire, wor, xnor, xor},
    morekeywords=[2]{`define, `undef, `ifdef, `ifndef, `else, `elsif, `endif, `include,
                    `resetall, `timescale, `celldefine, `endcelldefine, `default_nettype,
                    `unconnected_drive, `nounconnected_drive},
    morekeywords=[3]{\$display, \$write, \$strobe, \$monitor, \$time, \$stime, \$realtime,
                    \$finish, \$stop, \$dumpfile, \$dumpvars, \$dumpon, \$dumpoff, \$dumpall,
                    \$dumpflush, \$dumplimit, \$random, \$readmemb, \$readmemh},
    morecomment=[l]{//},
    morecomment=[s]{/*}{*/},
    morestring=[b]",
    sensitive=true,
    keywordstyle=\color{blue}\bfseries,
    keywordstyle=[2]\color{purple}\bfseries,
    keywordstyle=[3]\color{orange}\bfseries,
    stringstyle=\color{red},
    numberstyle=\tiny\color{gray},
    stepnumber=1,
    numbersep=10pt,
    backgroundcolor=\color{white},
    showspaces=false,
    showstringspaces=false,
    showtabs=false,
    frame=single,
    tabsize=4,
    captionpos=b,
    breaklines=true,
    breakatwhitespace=false,
    title=\lstname,
    escapeinside={\%*}{*},
    morekeywords={*,...}
}









% 轻松引用, 可以用\cref{}指令直接引用, 自动加前缀.
% 例: 图片label为fig:1
% \cref{fig:1} => Figure.1
% \ref{fig:1}  => 1
\usepackage[capitalize]{cleveref}
% \crefname{section}{Sec.}{Secs.}
\Crefname{section}{Section}{Sections}
\Crefname{table}{Table}{Tables}
\crefname{table}{Table.}{Tabs.}

% \setmainfont{Palatino Linotype.ttf}
% \setCJKmainfont{SimHei.ttf}
% \setCJKsansfont{Songti.ttf}
% \setCJKmonofont{SimSun.ttf}
\punctstyle{kaiming}
% 偏好的几个字体, 可以根据需要自行加入字体ttf文件并调用

\renewcommand{\emph}[1]{\begin{kaishu}#1\end{kaishu}}

% 对 section 等环境的序号使用中文
% \renewcommand \thesection{\zhnum{section}、}
% \renewcommand \thesubsection{\arabic{section}}


%%%%%%%%%%%%%%%%%%%%%%%%%%%
%改这里可以修改实验报告表头的信息
\newcommand{\name}{裴晨皓\ 竹彦博\ 纪弘璐}
\newcommand{\studentNum}{2023K8009916003\ \ \ \ 2023K8009916001\ \ \ \ 2023K8009916002}
\newcommand{\major}{计算机科学与技术}
\newcommand{\labNum}{Project 7}
\newcommand{\labName}{Cache设计专题实验}
%%%%%%%%%%%%%%%%%%%%%%%%%%%

\begin{document}

\input{tex_file/head.tex}

%%%%%%%%%%%2个图
%%%%%%%%%%%%%%%%%%%%%%%%%%%%%%%%%%%%%%%%%%%%%%%%%%%%%%%%%%%%%5
\section{实验简介}

这次实验中,本组同学按照讲义提示,先实现了独立的Cache模块,又向其中加入了非缓存访问的处理逻辑,作为ICache和DCache分别接入流水线;同时修改了AXI转接桥逻辑,使其接口与Cache面向AXI转接桥的接口相匹配,并加入了突发传输逻辑;此外,还对MMU模块进行修改,利用其对存储访问类型进行判定,从而决定Cache模块内走可缓存/非缓存的处理逻辑;最后,进一步修改Cache和流水线模块,使其支持CACOP指令的处理。


\section{设计方案介绍}
下图是完成本次实验后的处理器结构框图:

\begin{figure}[H]
    \centering
    \includegraphics[width=0.65\textwidth]{fig/结构框图.png}
    \caption{处理器结构框图}
\end{figure}

\subsection{总体思路}

本次实验新增加了Cache模块,上面的结构框图还不够直观,下图直观展现新增Cache后各主要模块的协同工作情况:

\begin{figure}[H]
    \centering
    \includegraphics[width=0.58\textwidth]{fig/abstract_logic.png}
    \caption{协同工作逻辑图}
\end{figure}

\subsection{Cache模块设计}

% 在讲义给出的已有状态转移逻辑的基础上,引入对非缓存情况的处理以及CACOP指令后,Cache模块的状态转移图如下所示:

% \subsection{主状态机}
% 首先是对Cache查找请求的接收

% \subsection{Write\ Buffer状态机}

\subsubsection{对Cache请求的接收}
按照讲义说明,空闲的IDLE状态和LOOKUP状态都是可以接收流水线发来的请求的状态。如果主状态机处于这两个状态,并且在流水线发来Cache请求/CACOP指令时,只要不存在冲突(阻塞),Cache就可以接收它。

具体的冲突(阻塞)情况如下:

\begin{figure}[H]
    \centering
    \includegraphics[width=0.95\textwidth]{fig/stall.png}
    \caption{来自流水线的Cache请求-冲突(阻塞)情况(单行代码较长, 故采用截图)}
\end{figure}

\begin{enumerate}
    \item 数据冲突:
    
    如果正在处理一条写命中Cache请求:
    \begin{itemize}
        \item 若新的正常访存请求地址与正在处理的请求地址存在写后读冲突,那么需要阻塞。
        \item 若新发来的请求为CACOP,也需要阻塞。因为CACOP指令与命中的store指令一样,涉及对Dirty位的改写,因此需要保证它们被顺序处理(顺序改写Dirty位),这里选择在store处理完后再让CACOP“进来”。
    \end{itemize}

    \item 结构冲突:
    
    \begin{itemize}
        \item 根据讲义提示,若Write\ Buffer状态机正处于WRITE状态,说明正在处理对Cache的写入,若流水线再发来一个Load类的Cache访问请求,与正在处理的写请求访问的是同一个Bank,则由于Bank的实现使用的是单端口RAM,故不能马上接收这一新请求。
        \item 还是由于单端口RAM的特性,我们的设计中,RAM读写共用一个地址信号,在Write时这一地址信号被设为Write\ Buffer里保存的地址,而非新请求的地址,因此新请求的RAM地址与Write\ Buffer里保存的地址不同时,应当阻塞新请求,否则会从错误的RAM地址做读取。
    \end{itemize}
    
    %%%%%%%%%%%%%%%%%%%%%%%%%%%%%%%%%%%%%CACOP
    %%%%%%%%%%%%%%%%%%%%%%%%%%%%%%%%%%%%%CACOP
    %%%%%%%%%%%%%%%%%%%%%%%%%%%%%%%%%%%%%CACOP
    %%%%%%%%%%%%%%%%%%%%%%%%%%%%%%%%%%%%%CACOP
\end{enumerate}

进一步地,我们可以得到Cache接收新请求的条件,即addr_ok/cacop_ok的更新逻辑:
\begin{lstlisting}[language=Verilog, caption={对Cache请求的接收-addr_ok/cacop_ok}]
assign addr_ok = !cacop && ((m_current_state == M_IDLE && !stall) || ((m_current_state == M_LOOKUP) && hit && !cacop_hit_inv_reg && valid && !stall));
assign cacop_ok = (m_current_state == M_IDLE && !stall) || ((m_current_state == M_LOOKUP) && hit && !cacop_hit_inv_reg && cacop && !stall);
\end{lstlisting}

\begin{itemize}
    \item 在没有阻塞的情况下,若是处在IDLE或是有正常访存hit的LOOKUP阶段,则可以接受请求。
    \item 在LOOKUP阶段有可能存在hit的CACOP指令,但由于后续状态机还要进入MISS等阶段做Cache的写回,还要用到Request\ Buffer里保存的信息,故这时不能接受新请求。
    \item 在我们的设计中,为了保障性能,减少CACOP造成的指令重取开销,所以在CACOP指令访问icache和IF级访问icache(取指)同时发生时,优先相应CACOP,拉高cacop_ok,拉低addr_ok。
\end{itemize}

接收请求的时候,更新Request\ Buffer:
\begin{lstlisting}[language=Verilog, caption={对Cache请求的接收-Request\ Buffer更新(时序逻辑块代码片段节选)}]
else if(((m_current_state == M_IDLE) && cacop && !stall) || ((m_current_state == M_LOOKUP) && hit && !cacop_hit_inv_reg && cacop && !stall)) begin
    op_reg <= 1'b0;
    cached_reg <= 1'b1;
    tag_reg <= cacop_tag;
    index_reg <= cacop_index;
    offset_reg <= cacop_offset;
    wstrb_reg <= 4'b0;
    wdata_reg <= 32'b0;
    cacop_st_tag_reg <= cacop_st_tag;
    cacop_idx_inv_reg <= cacop_idx_inv;
    cacop_hit_inv_reg <= cacop_hit_inv;
end
else if(((m_current_state == M_IDLE) && valid && !stall) || ((m_current_state == M_LOOKUP) && hit && !cacop_hit_inv_reg && valid && !stall)) begin
    op_reg <= op;
    cached_reg <= cached;
    tag_reg <= tag;
    index_reg <= index;
    offset_reg <= offset;
    wstrb_reg <= wstrb;
    wdata_reg <= wdata;
    cacop_st_tag_reg <= 1'b0;
    cacop_idx_inv_reg <= 1'b0;
    cacop_hit_inv_reg <= 1'b0;
end
\end{lstlisting}

若是CACOP指令,则向Request\ Buffer存入CACOP指令的具体类型、index、tag等,并把存储访问类型置为可缓存(因为CACOP访问的就是Cache本身);若是正常访存,则把正常访存对应的tag、index等信息保存下来,并且根据实际情况设定可缓存/非缓存。



\subsubsection{正常访存Cache命中情况的处理}
主状态机在LOOKUP状态对命中进行判定,若这一周期没有新的请求或是请求被阻塞,则回到IDLE空闲状态;若有未被阻塞的新请求,则下一拍仍在LOOKUP,对新请求进行处理。

在可缓存时(Cacop采用查询索引时也复用此逻辑),针对每一路Cache,对同一index,若请求的tag与Cache中存储的tag相同,且cache行有效,那么就为命中:
\begin{lstlisting}[language=Verilog, caption={命中判定}]
wire hit_way_1 = (tagv1_rdata[20: 1] == tag_reg && tagv1_rdata[0]) && (cached_reg || cacop_hit_inv_reg);
wire hit = hit_way_0 || hit_way_1;
\end{lstlisting}

读命中比较简单,根据命中的是哪一路,选择相应的数据传给流水线即可。

若是写命中,要把写相关信息以及命中哪一路的信息保存在Write Buffer,并且还要把命中的Bank内相应index的位置上原来的“字”暂存下来,然后与流水线发来的一个字大小的写数据通过wstrb掩码进行合并、拼接,得到最终写回Cache的数据:

\begin{lstlisting}[language=Verilog, caption={写命中情况下最终写回Cache的数据}]
wire [31: 0] hit_mask = {{8{w_wstrb_reg[3]}}, {8{w_wstrb_reg[2]}}, {8{w_wstrb_reg[1]}}, {8{w_wstrb_reg[0]}}};
wire [31: 0] hit_wdata = (w_wdata_reg & hit_mask) | (w_prev_data & ~hit_mask);
\end{lstlisting}

在Write\ Buffer状态机处于WRITE状态时,即可将数据写回Cache,并更新Dirty位。


\subsubsection{正常访存Cache未命中情况的处理}
发现未命中后,主状态机会由LOOKUP状态转到MISS状态,做Cache替换。被替换的Cache行若Dirty位为1,则先拉高wr_req请求把数据写回内存,等待AXI转接桥能够接收后(wr_rdy拉高),跳转到REPLACE。

在REPLACE拉高rd_req,请求把新行从内存通过突发传输读出来,待AXI转接桥能够接收请求后(rd_rdy拉高),跳转到REFILL状态,接收内存突发传回的数据,并存入Cache的data_ram里。

不难想到,内存传回的16B位于不同的Bank里,故此处设立一计数器,每接收到一个字,计数器就增1,用于辅助把数据写入不同的Bank(传完后,last拉高时要重置,以便下次传输):

\begin{lstlisting}[language=Verilog, caption={突发读-计数器}]
always @(posedge clk) begin
    if (rst) begin
        read_cnt <= 2'b0;
    end
    else if (ret_last == 1'b1) begin
        read_cnt <= 2'b0;
    end
    else if (ret_valid) begin
        read_cnt <= read_cnt + 2'b1;
    end
end
\end{lstlisting}

由于传回的第1个字写入被替换路的第0个Bank,第2个字写入第1个Bank,以此类推,因此根据计数器设计如下信号,生成Bank选择的独热码——第几个Bank被选中时,对应位为1,其余位为0:

\begin{lstlisting}[language=Verilog, caption={Bank选择信号}]
wire [3: 0] data0_wbank_sel = (m_current_state == M_REFILL) ? (4'b1 << read_cnt) : (4'b1 << w_offset_reg[3: 2]);
\end{lstlisting}

(上面的w_offset_reg为写命中时的Bank选择,类似地,也是用其生成独热码。)

当命中(选中)第0号Bank时,如果第0路写控制信号拉高(即REFILL状态下第0路被替换),那么就把这一个Bank的字节写使能拉高,否则置全0,其它Bank与此类似:
\begin{lstlisting}[language=Verilog, caption={写Bank选择信号转换为写使能信号}]
assign data0_bank0_we = ((data0_wbank_sel == 4'h1) && data0_we) ? 4'b1111 : 4'b0;
\end{lstlisting}

上面提到的第0路写控制信号用于控制“这一拍是否要对第0路做写操作”,具体如下:

\begin{lstlisting}[language=Verilog, caption={第0路写控制信号}]
wire data0_we = !cacop_reg && cached_reg 
            && (((m_current_state == M_REFILL) && (replace_way == 1'b0) && ret_valid) 
             || ((w_current_state == W_WRITE) && (w_way_reg == 1'b0) && w_we_reg));
\end{lstlisting}

data0_we拉高的条件为:在非缓存情况下,在REFILL状态重填第0路,且数据从AXI转接桥发回(ret_valid);在写命中第0路的情况下,Write\ Buffer处于Write状态。

还要注意的是,store操作未必是对整个字的修改,所以在写未命中情况的REFILL状态下,重填被store修改的那个字时,要特别处理:
\begin{lstlisting}[language=Verilog, caption={写未命中-数据拼接}]
wire [31: 0] refill_mask = {{8{wstrb_reg[3]}}, {8{wstrb_reg[2]}}, {8{wstrb_reg[1]}}, {8{wstrb_reg[0]}}};
wire [31: 0] refill_wdata = (ret_data & ~refill_mask) | (wdata_reg & refill_mask);
\end{lstlisting}
首先根据wstrb生成按bit的掩码mask,然后用掩码把AXI转接桥传回的完整字的相应部分改成store指令写入的值即可。

这时,结合前面几种情况,我们可以得出向Cache中写入的数据的选择逻辑:
\begin{lstlisting}[language=Verilog, caption={写入Cache的数据选择逻辑}]
assign data0_wdata = (w_current_state == W_WRITE) ? hit_wdata : 
                     !op_reg ? ret_data : 
                     (read_cnt == offset_reg[3: 2]) ? refill_wdata : ret_data;
\end{lstlisting}
这三个条件依次对应于:写命中、读未命中(需要写入AXI转接桥发回的ret_data)、写未命中(如果刚好是CPU发来的store指令修改的那个字,要向Cache写入拼接后的内容refill_wdata,其他情况直接填入AXI转接桥返回的字)

此外,这里还要注意,发给内存的写数据是整行一次发回的,而从内存读回的数据是逐字接收的。
\subsubsection{非缓存情况的处理}
非缓存的情况下,虽然访存请求会经过Cache模块,但并不会与Cache内存储的信息进行交互。这种情况状态机的转换路径为:主状态机在LOOKUP接收请求后直接进入MISS,然后分读写两种情况:

\begin{enumerate}
    \item 若是读,则下一拍直接进入REPLACE状态向AXI转接桥发读请求,握手成功后进入REDILL,收到数据后回到IDLE。
    \item 若是写,则向AXI转接桥发送写请求,握手成功后进入REPLACE,等待AXI转接桥发来数据写入成功信号data_ok(wr_complete)后,回到IDLE。
\end{enumerate}
这里要注意的是,非缓存一定要等数据真正写入内存,才能回IDLE接收新的请求,在此之前的新请求都需要被阻塞。从而保证了使用非缓存访问IO外设的正确性。

这时,我们就能写出完整的发给流水线的data_ok信号逻辑:
\begin{lstlisting}[language=Verilog, caption={data_ok信号逻辑}]
assign data_ok = !cacop_reg && (((m_current_state == M_LOOKUP) && hit) || 
        (m_current_state == M_REFILL) && !cached_reg && (op_reg == 1'b0) && ret_valid && ret_last || 
        (m_current_state == M_REPLACE) && !cached_reg && (op_reg == 1'b1) && wr_complete || 
        (m_current_state == M_REFILL) && cached_reg && ret_valid && (read_cnt == offset_reg[3: 2]));
\end{lstlisting}
上述代码中:
\begin{enumerate}
    \item 第一行为缓存命中,直接发data_ok;第4行为未命中重填,访存指令访问的那一个字的数据从AXI转接桥发回(根据计数器和offset判断),就可以认为获得了load指令需要的字(的部分)/store指令修改的字,于是也可以发data_ok。
    \item 第二、三行是非缓存情况,与前文描述一致,load指令拿到数据/store指令真正完成写入(wr_complete),就发出data_ok。
\end{enumerate}

此外还要注意,可缓存情况下向内存发送的请求地址是按Cache行对齐的,可能与流水线发给Cache的地址不同;而非缓存情况下则是按字对齐(4字节),与流水线发来的地址相同——二者要做区分,以读请求地址为例:
\begin{lstlisting}[language=Verilog, caption={内存读请求地址}]
assign rd_addr = cached_reg ? {tag_reg, index_reg, 4'b0000} : {tag_reg, index_reg, offset_reg};
\end{lstlisting}


\subsubsection{CACOP指令的添加}
CACOP指令共有3种,有2种采用的都是地址直接索引,还有1种使用查询索引,使用查询索引的这种与普通load指令类似,需要对命中与否进行判断。这3种指令的共同点是,如果操作的相应Cache行是脏的,需要先将其写回内存(手册未说明Store\ Tag这种情况是否要写回脏行,我们的设计中采取了写回的方式处理)。

在状态转移方面,使用查询索引的方式若未命中,则没有任何后续处理,直接回到IDLE;其他情况则一律进入MISS。若处理的是脏行,则复用替换时写回脏行的逻辑,需要拉高wr_req,向AXI转接桥发送写请求。待AXI接口发回的wr_rdy拉高,能够接收写请求后进入REPLACE状态,在REPLACE状态等待数据完成写入(wr_complete拉高)后,在REFILL状态对tag/v位进行处理,最后回到IDLE。

CACOP的写回逻辑复用了未命中做替换时的逻辑,所以相当于也“替换”走了某一行,所以复用原有的replace_way逻辑(在原有基础上做添加):

\begin{lstlisting}[language=Verilog, caption={“替换(写回内存)”路选择}]
else if ((m_current_state == M_LOOKUP) && (!hit || hit && cacop_hit_inv_reg)) begin
    if(cacop_st_tag_reg || cacop_idx_inv_reg) begin
        replace_way <= offset_reg[0];
    end
    else if(cacop_hit_inv_reg) begin
        replace_way <= hit_way_0 ? 1'b0 : 1'b1;
    end
    else begin
        replace_way <= rand_way;
    end
end
\end{lstlisting}
若是采用地址直接索引的两类CACOP,则根据偏移量offset的最低位选择操作哪一路,所以写回内存的与它是同一路;若是采用的查询索引方式,则操作/写回的是命中的那一路。

根据前面的描述:CACOP要根据具体操作的那一路,产生tagv0/1的写使能,清零tag/v位——于是得到下面的逻辑:
\begin{lstlisting}[language=Verilog, caption={tagv写逻辑(以第0路为例)}]
assign tagv0_we 
= (cacop_st_tag_reg || cacop_idx_inv_reg) ? ((m_current_state == M_REFILL) && !offset_reg[0]) :
  (cacop_hit_inv_reg) ? ((m_current_state == M_REFILL) && (replace_way == 1'b0)) :
  (cached_reg && (replace_way == 1'b0) && ret_last); //可缓存,从内存获取完重填数据
assign tagv0_wdata = cacop_reg ? 32'd0 : {tag_reg, 1'b1};
\end{lstlisting}
类似地,相应dirty位也要做修改(归零),此处不再赘述。

此外,综合前面的所有情况,我们可以得到发给内存的wstrb:
\begin{lstlisting}[language=Verilog, caption={发给内存的wstrb}]
assign wr_wstrb = (cached_reg || cacop_reg) ? 4'b1111 : wstrb_reg;
\end{lstlisting}
只有非缓存情况下才要发送Request\ Buffer里保存的值;其余情况都是整行写回,即每一个字都要完整写回,所以wstrb置为全1。


\subsection{对类SRAM-AXI转接桥的修改}
这部分修改较为简单,只需要让AXI转接桥支持突发传输;并且修改原来的类SRAM接口,使转接桥接口与Cache模块面向AXI转接桥的接口相匹配,然后使转接桥顶层仲裁、分发逻辑与修改后的接口相匹配即可。

\subsubsection{对顶层模块的修改}
原先对读写请求的区分,要先通过sram_req判断有无请求,再根据sram_wr区分是读还是写,而现在Cache方面会直接发来明确的读写请求,所以要对部分逻辑做相应更改。

以ar_id为例:
\begin{lstlisting}[language=Verilog, caption={ar_id-原}]
assign ar_id = (~sram_wr_2 && sram_req_2) ? 2'b10 : (~sram_wr_1 && sram_req_1) ? 2'b01 : 2'b00;
\end{lstlisting}
\begin{lstlisting}[language=Verilog, caption={ar_id-现}]
assign ar_id = rd_req_d ? 2'b10 : rd_req_i ? 2'b01 : 2'b00;
\end{lstlisting}
(ar_id为顶层用于仲裁的独热码,10表示处理访存级dcache的请求,01表示处理取值级icache的请求,00表示无请求。并且,两个流水级请求同时到来时,为保障性能,先处理访存级dcache的请求。)

为了支持突发,要引入len信号,表示突发传输的长度:
\begin{lstlisting}[language=Verilog, caption={ar_len信号}]
assign ar_len = ar_id[1] ? ((rd_type_d == 3'b100) ? 8'd3 : 8'd0) : ((rd_type_i == 3'b100) ? 8'd3 : 8'd0);
\end{lstlisting}
这行代码的type信号由Cache发来,是这次请求传输的字节数以2为底的对数(3'b100表示16字节,3'b010表示4字节)。所以最终长度的生成逻辑为:先跟据回应/处理请求的独热码ar_id做仲裁,对取指icache/访存级dcache进行选择,然后根据相应type,把len设为3(传输4次,单次传4字节,共16字节)或0(传输1次,单次传4字节,共4字节)。aw_len与此类似,不过由于取指级不会发写请求所以省去了仲裁,更加简单,不再赘述。

% 由于Cache发来的Cache行写回请求是一次性发给转接桥16B数据,所以要把输入的wdata扩展到128位;Cache发来的读请求都是逐字接收,所以不需要修改rdata的宽度。进一步考虑:
此外,由于Cache发来的写请求是一次性发给转接桥所有数据,所以要把输入的wdata扩展到128位以应对整行写回;而Cache发来的读请求,在接收数据时,都是逐字接收,所以不需要修改rdata的宽度。进一步考虑:
\begin{itemize}
    \item 逐字接收时,AXI要给Cache发送last信号,表示当前传输的是最后一个字;
    \begin{lstlisting}[language=Verilog, caption={last信号}]
    assign ret_last_i = r_id[0] & r_last;
    assign ret_last_d = r_id[1] & r_last;
    \end{lstlisting}
    根据仲裁独热码,把读响应通道的last发给icache/dcache即可。
    \item 一次性写回16B时,AXI顶层模块会把它完整的转交给写请求/数据通道模块,不在顶层进行处理。那么写操作last信号也就不用在顶层处理了,只要写请求/数据通道模块内在完整传输的最后一次写时拉高last,用顶层的AXI接口传给内存即可。此外,显然这一信号也不必传给Cache。
\end{itemize}




\subsubsection{写请求/数据/响应通道模块}
此处需要增加设计,便于实现写突发传输。具体地,首先要加一个计数器,记录已经传输了多少个字:
\begin{lstlisting}[language=Verilog, caption={写通道-计数器}]
always @(posedge clk) begin
    if (!resetn) begin
        counter <= 8'd0;
    end
    else if (wready && wvalid && wlast) begin
        counter <= 8'd0;
    end
    else if (wready && wvalid) begin
        counter <= counter + 8'd1;
    end
end
\end{lstlisting}
每完成一次写数据握手,计数器就加1;当last拉高时说明已经是最后一次传输,于是把计数器清零,以便下次传输。

一次传输的数据可能长达16B,但每次传输只能发4B,所以要利用counter计算偏移量,选择当前要发给内存的数据:
\begin{lstlisting}[language=Verilog, caption={wdata信号}]
assign wdata = data_reg[counter * 32 +: 32];
\end{lstlisting}

在计数器数值(已完成传输的数量)与本次写操作的len相等时,说明本次写操作的最后一个写数据已经发出,于是拉高last信号:
\begin{lstlisting}[language=Verilog, caption={wlast信号}]
assign wlast = (counter == len_reg);
\end{lstlisting}

此外,还要对状态机的转移条件进行修改,由于突发传输是一次请求握手,多次数据握手,所以:原先以“写数据通道握手成功”为条件进行转移的路径,要把转移条件改为“写数据通道数据传输完毕”,即向其中加入wlast信号:
\begin{lstlisting}[language=Verilog, caption={修改后的状态转移逻辑(局部)}]
BUSY: begin
    if (awready && awvalid && wready && wvalid && wlast) begin
        next_state = IDLE;
    end
    else if (awready && awready) begin
        next_state = AW_FIRE;
    end
    else if (wready && wvalid && wlast) begin
        next_state = W_FIRE;
    end
    else begin
        next_state = BUSY;
    end
end
AW_FIRE: begin
    if (wready && wvalid && wlast) begin
        next_state = IDLE;
    end
    else begin
        next_state = AW_FIRE;
    end
end
\end{lstlisting}

\subsubsection{读请求/响应通道模块}
这部分修改较为简单:
\begin{itemize}
    \item 读请求通道接收并保存请求信息时,要额外保存len信号,从而作为arlen发给内存;
    \item 读响应通道接收内存发来的读数据时,要额外保存last信号,从而传给Cache模块,表示当前传输的是否为最后一个字。
\end{itemize}



\subsection{对流水线的修改-加入CACOP指令支持}
由于使用查询索引方式的那一类CACOP指令的VA可被当做一条普通load指令访问Cache,所以我们的CACOP与load访存指令放在同一级实现(MEM级)。此外,由于操作指令 Cache 的 CACOP 指令和取指有特权资源相关冲突,所以要仿照上一章,新增一种重取机制。

CACOP指令操作类型有3种,操作的对象有icache和dcache这2种,共计六种情况,每一种情况有单独的信号发给Cache,拉高即表明发送相应请求:
\begin{lstlisting}[language=Verilog, caption={CACOP向Cache发送的请求(以对icache的store\ tag类型为例)}]
assign cacop_st_tag_i  = in_valid && cacop && !cacop_i_fire && (inst_4_0[2:0] == 3'b000) && (inst_4_0[4:3] == 2'b00) && !this_flush && !this_tlb_flush && !this_cacop_flush;
\end{lstlisting}
以对icache的store\ tag类型为例:若当前MEM级为CACOP指令,且根据指令中的code(inst_4_0)判断为操作icache的store\ tag类指令,并且请求没有被接收(没有握手成功,cacop_i_fire为0),且当前指令后续不需要被冲刷,那么就持续拉高cacop_st_tag_i,向Cache发请求。

在CACOP指令收到Cache发回的cacop_ok信号后,拉高cacop_i_fire,表示针对icache的这条指令握手成功:
\begin{lstlisting}[language=Verilog, caption={cacop_i_fire信号}]
always @(posedge clk) begin
    if(rst) begin
        cacop_i_fire <= 1'b0;
    end
    else if(in_valid && ready_go && out_ready) begin
        cacop_i_fire <= 1'b0;
    end
    else if(cacop_req_i && cacop_ok_i) begin
        cacop_i_fire <= 1'b1;
    end
end
\end{lstlisting}
(这条指令从MEM级流出时,cacop_i_fire要清零,以便下次使用。)

由于CACOP指令也需要握手,所以没握手成功时要被阻塞在MEM级:
\begin{lstlisting}[language=Verilog, caption={ready_go信号修改(省略原有内容)}]
assign ready_go = ...
!(cacop && (inst_4_0[2:0] == 3'b000) && !(|mmu_ecode_d) && !(cacop_req_i && cacop_ok_i || cacop_i_fire)) &&
!(cacop && (inst_4_0[2:0] == 3'b001) && !(|mmu_ecode_d) && !(cacop_req_d && cacop_ok_d || cacop_d_fire));
\end{lstlisting}
即:如果不是“既没发生TLB异常,又没握手成功”CACOP指令,才能继续流动。

由于采用查询索引方式的那类CACOP指令(cacop_hit_inv_i/d)需要进行虚实地址翻译,可能出现TLB异常,所以要把它加到异常传递(检测)逻辑中:
\begin{lstlisting}[language=Verilog, caption={修改后的异常传递逻辑}]
else if (in_valid && ready_go && out_ready) begin
    has_exception_out <= has_exception || ((|mmu_ecode_d) & (res_from_mem || mem_we || cacop_hit_inv_i || cacop_hit_inv_d));
end
\end{lstlisting}

关于CACOP指令造成的的重取,基本完全仿照之前TLB指令导致重取的机制实现。由于MEM级实现CACOP,所以其流入下一级(RDW)后进行“提交”,从而冲刷前面的流水级,并把取指PC设为CACOP指令下一条顺序指令的PC(PC加4)。具体细节不再赘述。


\subsection{MMU模块对存储访问类型的判定}
以取指级的情况为例:
\begin{lstlisting}[language=Verilog, caption={mat_i信号生成逻辑}]
assign mat_i = (crmd_da_value && !crmd_pg_value) ? crmd_datf_value :
               (inst_sram_vaddr[31: 29] == dmw0_vseg_value && dmw0_plv_cond) ? dmw0_mat_value :
               (inst_sram_vaddr[31: 29] == dmw1_vseg_value && dmw1_plv_cond) ? dmw1_mat_value :
                tlb_s0_mat;
\end{lstlisting}
(mat为0时表示非缓存,1表示可缓存。)
\begin{itemize}
    \item 直接地址翻译模式下,MAT由 CSR.CRMD 的 DATF 域决定;
    \item 直接映射地址翻译模式下,MAT由所命中的直接映射窗口中的MAT 配置信息(dmw0/1_mat_value)决定;
    \item 页表映射地址翻译模式下,MAT由虚实地址转换所用页表项中的 MAT 域配置信息决定。
\end{itemize}

\section{Debug记录}

\subsection{忘记接入存储访问类型信号导致向外设发起突发请求}
在接入dcache进行调试时,发现如下图所示波形:
\begin{figure}[H]
    \centering
    \includegraphics[width=0.95\textwidth]{fig/外设突发.png}
    \caption{忘记接入存储访问类型信号导致向外设发起突发请求-错误波形}
\end{figure}
上图中光标所在位置处,一次读请求握手成功,arlen为3,arid为1,是一次dcache发起的突发读请求(读4次,每次4B)。然而,在红框位置处,收到第一个读数据时(rvalid拉高),rlast居然也拉高了,相当于后续3个4B没有传回就结束了,不禁令人困惑。

后续检查发现,我们忘记接入存储访问类型的信号,导致所有访存请求都被当做了可缓存访问。上图波形中原本应为非缓存的情况,恰巧arlen为3,导致误企图向外设发起突发请求,而外设并不支持突发传输,故只传回了第一个4B数据,后续数据没有传回。

正确接入存储访问类型信号后,问题得以解决。

\subsection{两条连续写命中指令导致同一Bank同一Cache行的“写后读”}

对于一条写命中指令,其第0个周期会发送请求,第1个周期会得到命中结果,并读出相应Bank、相应Cache行的原有数据,第2个周期会得出写入值与原值根据wstrb拼接后的值,在第2个周期进入第3个周期的时钟上升沿写回Cache。最初的设计中,连续两条针对同一Bank、同一Cache行的写命中指令出现下图所示情况:
\begin{figure}[H]
    \centering
    \includegraphics[width=0.95\textwidth]{fig/连续写命中.png}
    \caption{两条连续写命中指令导致同一Bank同一Cache行的“写后读”-错误波形}
\end{figure}
红框位置是第1条写命中指令发来的请求,请求写入850813af,wstrb为1111;黄框位置是第2条写命中指令发来的请求,请求写入40404040,wstrb为0100,即只把第2字节拼接到Cache内原来存储的字里,两条指令命中的是同一Bank、同一Cache行——按照正常时序,最终第二条指令写回Cache的字应为854013af。

然而,在蓝框所示位置处,第1条指令才刚刚得到其应当写入Cache的值,而此时第2条指令也已经读出了命中Bank的数据,由于第1条指令要在进入下一周期的上升沿才写回Cache,所以第2条指令相当于读出了Cache的旧值,出现了“写后读”。毫不意外地,在下个周期,第2条指令用Cache的旧值与自己写入的字节进行拼接,得出了要写回Cache的00400000,而非正确的854013af。

最终解决方式为引入阻塞,从而避免这种表面“连续写”,本质暗含“写后读”的情况出现。

\subsection{Cache阻塞时忘记拉低addr_ok}

我们设计的stall信号是用于阻塞Cache的,即拉高stall时不应接收新请求,所以发向流水线的addr_ok信号也应当拉低,而最初我们忘了用stall信号控制addr_ok,导致下图情况:

\begin{figure}[H]
    \centering
    \includegraphics[width=0.95\textwidth]{fig/阻塞stall.png}
    \caption{Cache阻塞时忘记拉低addr_ok-错误波形}
\end{figure}

在白框所示位置处,stall拉高,而同一周期内,红框位置处却拉高了addr_ok,与流水线发来的请求“误握手”成功。导致流水线方认为请求已被接收,只需要等待后续data_ok,而Cache方却并不认为自己接收了这次请求,自然也不会给流水线发data_ok信号。由于流水线一直等不到data_ok,后续指令也就都被阻塞,Cache也收不到新请求,整个系统卡死(如黄框所示,Cache的状态恒处于IDLE;图片最上方的PC值也不再改变)。

解决方法十分简单,给原有的addr_ok信号“与”上“!stall”,限制stall拉高时对请求的接受即可。


\subsection{突发传输请求地址忘记按Cache行对齐}

\begin{figure}[H]
    \centering
    \includegraphics[width=0.95\textwidth]{fig/突发未对齐.png}
    \caption{突发传输请求地址忘记按Cache行对齐-错误波形}
\end{figure}

如上图所示,Cache向内存发读请求时,若是可缓存情况下做突发传输,应该把流水线发来的地址按Cache行对齐再发给内存;而非缓存情况直接使用流水线发来的地址即可。在加入非缓存处理逻辑时,我们误把原有的突发传输请求地址先按Cache行对齐再发送的逻辑,直接改成了发送流水线传来的地址,进而在未按行对齐地址发起突发传输,取回了错误的一行数据。

正确做法是使用选择器将两种请求地址分开:
\begin{lstlisting}[language=Verilog, caption={发给内存的读请求地址}]
assign rd_addr = cached_reg ? {tag_reg, index_reg, 4'b0000} : {tag_reg, index_reg, offset_reg};
\end{lstlisting}


\section{合作说明}

本实验由本组成员共同合作完成,组内同学同等贡献。


\end{document}