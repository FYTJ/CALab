% 这是中国科学院大学计算机科学与技术专业《计算机组成原理(研讨课)》使用的实验报告 Latex 模板
% 本模板与 2024 年 2 月 Jun-xiong Ji 完成, 更改自由 Shing-Ho Lin 和 Jun-Xiong Ji 于 2022 年 9 月共同完成的基础物理实验模板
% 如有任何问题, 请联系: jijunxoing21@mails.ucas.ac.cn
% This is the LaTeX template for report of Experiment of Computer Organization and Design courses, based on its provided Word template.
% This template is completed on Febrary 2024, based on the joint collabration of Shing-Ho Lin and Junxiong Ji in September 2022.
% Adding numerous pictures and equations leads to unsatisfying experience in Word. Therefore LaTeX is better.
% Feel free to contact me via: jijunxoing21@mails.ucas.ac.cn

\documentclass[11pt]{article}

\usepackage[a4paper]{geometry}
\geometry{left=2.0cm,right=2.0cm,top=2.5cm,bottom=2.5cm}

\usepackage{ctex} % 支持中文的LaTeX宏包
\usepackage{amsmath,amsfonts,graphicx,subfigure,amssymb,bm,amsthm,mathrsfs,mathtools,breqn} % 数学公式和符号的宏包集合
\usepackage{algorithm,algorithmicx} % 算法和伪代码
\usepackage[noend]{algpseudocode} % 算法和伪代码
\usepackage{fancyhdr} % 自定义页眉页脚
\usepackage[framemethod=TikZ]{mdframed} % 创建带边框的框架
\usepackage{fontspec} % 字体设置
\usepackage{adjustbox} % 调整盒子大小
\usepackage{fontsize} % 设置字体大小
\usepackage{tikz,xcolor} % 绘制图形和使用颜色
\usepackage{multicol} % 多栏排版
\usepackage{multirow} % 表格中合并单元格
\usepackage{pdfpages} % 插入PDF文件
\usepackage{listings} % 在文档中插入源代码
\usepackage{wrapfig} % 文字绕排图片
\usepackage{bigstrut,multirow,rotating} % 支持在表格中使用特殊命令
\usepackage{booktabs} % 创建美观的表格
\usepackage{circuitikz} % 绘制电路图
\usepackage{zhnumber} % 中文序号(用于标题)
\usepackage{tabularx} % 表格折行

\definecolor{dkgreen}{rgb}{0,0.6,0}
\definecolor{gray}{rgb}{0.5,0.5,0.5}
\definecolor{mauve}{rgb}{0.58,0,0.82}
% \lstset{
%   frame=tb,
%   aboveskip=3mm,
%   belowskip=3mm,
%   showstringspaces=false,
%   columns=flexible,
%   framerule=1pt,
%   rulecolor=\color{gray!35},
%   backgroundcolor=\color{gray!5},
%   basicstyle={\small\ttfamily},
%   numbers=none,
%   numberstyle=\tiny\color{gray},
%   keywordstyle=\color{blue},
%   commentstyle=\color{dkgreen},
%   stringstyle=\color{mauve},
%   breaklines=true,
%   breakatwhitespace=true,
%   tabsize=3,
% }



\lstset{
    % basicstyle=\ttfamily\small,
    basicstyle=\ttfamily\footnotesize,
    keywordstyle=\color{blue},
    commentstyle=\color{gray},
    numbers=left,
    numberstyle=\tiny,
    breaklines=true,
    frame=single,
}

\lstdefinelanguage{Verilog}{
    morekeywords={always, and, assign, attribute, begin, buf, bufif0, bufif1, case, casex, casez,
                 cmos, deassign, default, defparam, disable, edge, else, end, endattribute,
                 endcase, endfunction, endmodule, endprimitive, endspecify, endtable, endtask,
                 event, for, force, forever, fork, function, highz0, highz1, if, ifnone, initial,
                 inout, input, integer, join, large, macromodule, medium, module, nand, negedge,
                 nmos, nor, not, notif0, notif1, or, output, parameter, pmos, posedge, primitive,
                 pull0, pull1, pulldown, pullup, rcmos, real, realtime, reg, release, repeat,
                 rnmos, rpmos, rtran, rtranif0, rtranif1, scalared, signed, small, specify,
                 specparam, strong0, strong1, supply0, supply1, table, task, time, tran, tranif0,
                 tranif1, tri, tri0, tri1, triand, trior, trireg, unsigned, vectored, wait, wand,
                 weak0, weak1, while, wire, wor, xnor, xor},
    morekeywords=[2]{`define, `undef, `ifdef, `ifndef, `else, `elsif, `endif, `include,
                    `resetall, `timescale, `celldefine, `endcelldefine, `default_nettype,
                    `unconnected_drive, `nounconnected_drive},
    morekeywords=[3]{\$display, \$write, \$strobe, \$monitor, \$time, \$stime, \$realtime,
                    \$finish, \$stop, \$dumpfile, \$dumpvars, \$dumpon, \$dumpoff, \$dumpall,
                    \$dumpflush, \$dumplimit, \$random, \$readmemb, \$readmemh},
    morecomment=[l]{//},
    morecomment=[s]{/*}{*/},
    morestring=[b]",
    sensitive=true,
    keywordstyle=\color{blue}\bfseries,
    keywordstyle=[2]\color{purple}\bfseries,
    keywordstyle=[3]\color{orange}\bfseries,
    stringstyle=\color{red},
    numberstyle=\tiny\color{gray},
    stepnumber=1,
    numbersep=10pt,
    backgroundcolor=\color{white},
    showspaces=false,
    showstringspaces=false,
    showtabs=false,
    frame=single,
    tabsize=4,
    captionpos=b,
    breaklines=true,
    breakatwhitespace=false,
    title=\lstname,
    escapeinside={\%*}{*},
    morekeywords={*,...}
}









% 轻松引用, 可以用\cref{}指令直接引用, 自动加前缀.
% 例: 图片label为fig:1
% \cref{fig:1} => Figure.1
% \ref{fig:1}  => 1
\usepackage[capitalize]{cleveref}
% \crefname{section}{Sec.}{Secs.}
\Crefname{section}{Section}{Sections}
\Crefname{table}{Table}{Tables}
\crefname{table}{Table.}{Tabs.}

% \setmainfont{Palatino Linotype.ttf}
% \setCJKmainfont{SimHei.ttf}
% \setCJKsansfont{Songti.ttf}
% \setCJKmonofont{SimSun.ttf}
\punctstyle{kaiming}
% 偏好的几个字体, 可以根据需要自行加入字体ttf文件并调用

\renewcommand{\emph}[1]{\begin{kaishu}#1\end{kaishu}}

% 对 section 等环境的序号使用中文
% \renewcommand \thesection{\zhnum{section}、}
% \renewcommand \thesubsection{\arabic{section}}


%%%%%%%%%%%%%%%%%%%%%%%%%%%
%改这里可以修改实验报告表头的信息
\newcommand{\name}{裴晨皓\ 竹彦博\ 纪弘璐}
\newcommand{\studentNum}{2023K8009916003\ \ \ \ 2023K8009916001\ \ \ \ 2023K8009916002}
\newcommand{\major}{计算机科学与技术}
\newcommand{\labNum}{Project 5}
\newcommand{\labName}{AXI 总线接口设计专题实验}
%%%%%%%%%%%%%%%%%%%%%%%%%%%

\begin{document}

\input{tex_file/head.tex}

\section{实验简介与总体思路}
本实验中,我们首先对流水线CPU进行改造,把访存接口改为类SRAM接口,并将五级流水线改为七级以实现请求和数据响应分离(利于并行处理提高性能),然后按照教材提示设计了具有四个状态机的类 SRAM-AXI 转接桥,最终完成了本次实验。


\section{设计方案介绍}
下图是完成本次实验后的处理器结构框图:

\begin{figure}[H]
    \centering
    \includegraphics[width=0.9\textwidth]{fig/结构框图.jpg}
    \caption{处理器结构框图}
\end{figure}

\subsection{流水线处理器本身的进一步改进}

\subsubsection{五级流水线拆分为七级流水线}
进入本次实验,访存操作引入了握手机制,一次访存操作要分为发送请求和等待数据这两次握手(实际上用类SRAM接口的CPU的第2次“握手”为单向的data_ok,并非严格意义的握手)。为了提高流水线效率,我们将访存的流水级都拆成两级,一级用于请求,一级用于等待数据,具体表现为:IF级拆分为IF(相当于讲义的pre-IF)和IW(Instruction Waiting,相当于讲义的IF),MEM级拆分为MEM和RDW(Read Data Waiting,这是组成原理研讨课时使用的名称,实际上由于load/store操作都要等待data_ok,故不仅是Read Data,Write Data时也要在这一级等待)。

\subsubsection{IF阶段的取指请求设计}

% 根据类SRAM接口的定义,由于指令内存不涉及写操作,所以wr置为0,与写相关的信号也置为全0即可;指令长度为4字节,所以size恒置为2'b10。这部分代码逻辑直白,不在此处展示。

% 接下来关注req信号的问题:对同一个PC而言,在没完成取指令请求握手时,req要一直拉高;完成握手后,req要拉低,避免重复取指。为此,我们设置了handshake_done信号,用于记录当前IF阶段的指令是否已经完成取指请求握手:
这部分要重点关注req信号的问题:对同一个PC而言,在没完成取指令请求握手时,req要一直拉高;完成握手后,req要拉低,避免重复取指。为此,我们设置了handshake_done信号,用于记录当前IF阶段的指令是否已经完成取指请求握手:

% \begin{lstlisting}[language=Verilog, caption={handshake_done信号的定义}]
%     always @(posedge clk) begin
%         if(rst) begin
%             handshake_done <= 1'b0;
%         end
%         else if((req && addr_ok) || out_ready) begin
%             handshake_done <= !out_ready;
%         end
%         else if(ex_flush || ertn_flush || br_taken) begin
%             handshake_done <= 1'b0;
%         end
%     end
% \end{lstlisting}
% 考虑以下情况:
% \begin{enumerate}
%     \item IF当前PC取指请求握手成功(req && addr_ok):若下一级IW允许流入(out_ready为1),则IF将会把PC改为下一条指令的PC并发送取指请求,handshake_done自然也对应下一条指令的握手情况,所以这时应赋值为0;若IW不允许流入(out_ready为0),则handshake_done应该拉高,表示当前指令的握手已经完成。
%     \item IF当前PC取指请求未握手成功,但IW允许流入
% \end{enumerate}

\begin{lstlisting}[language=Verilog, caption={handshake_done信号的定义}]
    always @(posedge clk) begin
        if(rst) begin
            handshake_done <= 1'b0;
        end
        else if(ready_go) begin
            handshake_done <= !out_ready;
        end
        else if(ex_flush || ertn_flush || br_taken) begin
            handshake_done <= 1'b0;
        end
    end
\end{lstlisting}

考虑以下情况:
\begin{enumerate}
    \item 在IF阶段工作完成(完成取指,ready_go为1)后,若IW不允许流入(out_ready为0),则要把handshake_done拉高,记录当前指令的取指请求握手已完成;若IW允许流入(out_ready为1),则接下来应当发下一条指令的取指请求,故handshake_done应该重置为0。
    \item 出现冲刷情况时,由于IF在当拍就把PC改为冲刷后的新PC,故handshake_done也应重置为0。
\end{enumerate}

到上面这步还不够,因为若出现冲刷,那么IF会在同一周期改为新的PC,它不能被冲刷前PC的握手情况干扰;此外,若冲刷前的PC对应请求已经握手成功,在执行流的角度看,这条指令将不会在后续的阶段被执行——根据这两点可以想到:这时的handshake_done是无效的。故结合三类冲刷情况,我们另设handshake_done_effective信号:

\begin{lstlisting}[language=Verilog, caption={handshake_done_effective信号}]
    wire handshake_done_effective = handshake_done && !ex_flush && !ertn_flush && !br_taken;
\end{lstlisting}

此外,对于分支指令跳转冲刷的情况,由于ID阶段分支指令可能存在load-to-branch情况(br_stall,在ID根据指令类型和已有的load_use_stall信号得出),分支判定结果无法立即得出,所以这时需要暂停取指请求,直到得到分支结果。

进一步地,req信号也就能得到定义:
\begin{lstlisting}[language=Verilog, caption={req信号定义}]
    assign req = !handshake_done_effective && !(br_stall && ID_in_valid);
\end{lstlisting}
从自然语言角度讲,即:当前PC没有有效的握手完成行为,且没有未完成的分支判断,就可以发取指请求。

除了req,取指PC也需要更多考虑。原先不存在握手和延迟的取指请求一拍就能完成,但现在大多数时候都需要多拍;又因为取指有时依赖冲刷信号,而WB阶段报出的异常冲刷和ID给出的分支跳转冲刷有时只能维持一拍,所以要把报出的异常信号暂存下来。以异常冲刷信号为例:

\begin{lstlisting}[language=Verilog, caption={异常冲刷信号暂存}]
    always @(posedge clk) begin
        if(rst) begin
            ex_flush_reg <= 1'b0;
        end
        else if(in_valid && ready_go && out_ready) begin
            ex_flush_reg <= 1'b0;
        end
        else if(ex_flush) begin
            ex_flush_reg <= 1'b1;
        end
    end
\end{lstlisting}
在当前指令流入IW时,重置ex_flush_reg为0;在WB报出异常时,把ex_flush_reg拉高,即暂存异常信号。这样一来,在报出冲刷的当拍可以依赖ex_flush本身,下一拍则借助ex_flush_reg,把二者结合得到:
\begin{lstlisting}[language=Verilog, caption={IF最终使用的异常冲刷信号}]
    wire ex_flush_preserved = ex_flush | ex_flush_reg;
\end{lstlisting}
下面是异常冲刷情况下,对异常处理入口PC的生成(也是按照上面的逻辑设计):
\begin{lstlisting}[language=Verilog, caption={IF异常处理入口PC}]
    wire [31:0] ex_entry_preserved = ex_flush ? ex_entry : ex_entry_reg;
\end{lstlisting}

其他信号也与上面类似——把nextpc生成逻辑中涉及的冲刷信号和冲刷后重新取指的PC,都暂存为xxx_preserved信号,也就保证了取指PC在发请求的过程中始终保持正确:
\begin{lstlisting}[language=Verilog, caption={nextpc}]
    assign nextpc       = ex_flush_preserved ? ex_entry_preserved :
                          ertn_flush_preserved ? ertn_entry_preserved :
                          br_taken_preserved ? br_target_preserved : seq_pc;
\end{lstlisting}

% \subsubsection{对“PC未流入IW就已收到data_ok”情况的处理}

\subsubsection{指令的接收与传递}

正常情况应为位于IW的PC收到data_ok和指令,这与“Instruction Waiting”的含义一致。若IW收到指令后被阻塞,根据教材的提示:若选择暂停IF的取指请求的方式,则性能较差,故我们选择另一种方式——IF正常发取指请求,若IF的PC在进入IW前就已经收到data_ok,则把它暂存下来,待IW允许流入时再传递下去。

% \begin{enumerate}
%     \item IW阶段收到data_ok信号,获得IW的PC对应的指令。
%     \item IW阶段早已收到指令,但由于阻塞,未能流入ID,
% \end{enumerate}

因此,IW向ID传递的指令可分为三种情况(在流入ID的那个上升沿把下面的wire信号赋给ID的流水寄存器):

\begin{lstlisting}[language=Verilog, caption={IW向ID传递的指令}]
    wire [31:0] inst_out_wire = inst_valid_from_IF ? inst_from_IF :
                                inst_valid ? inst :
                                data_ok ? rdata :
                                32'd0;
\end{lstlisting}

\begin{enumerate}
    \item IW的PC对应的指令在IF阶段就已经收到(inst_valid_from_IF),后来它又被传给IW,故IW要把它(inst_from_IF)继续传递;
    \item IW之前已经收到过data_ok并把指令暂存了下来(inst_valid),则传递这条暂存的指令(inst)。
    \item 当前这一拍收到data_ok,则直接把当前访存得到的指令(rdata)传递下去。
\end{enumerate}

IF阶段暂存指令的逻辑如下:
\begin{lstlisting}[language=Verilog, caption={IF对指令的暂存}]
    always @(posedge clk) begin
        if(rst) begin
            inst_valid <= 1'b0;
            inst <= 32'd0;
        end
        else if(ex_flush || ertn_flush || br_taken) begin
            inst_valid <= 1'b0;
            inst <= 32'd0;
        end
        else if(in_valid && ready_go && out_ready) begin
            inst_valid <= 1'b0;
            inst <= 32'd0;
        end
        else if(data_ok && !out_ready && (inst_valid_out || IW_inst_valid) && (~(|discard))) begin
            inst_valid <= 1'b1;
            inst <= rdata;
        end
    end
\end{lstlisting}
\begin{enumerate}
    \item 有冲刷时,要把暂存的指令清空;当前流水级向后流动时,也应如此。
    \item 当IF被阻塞(!out_ready),且IW已经收到过来自IF或是直接来自内存的指令(inst_valid_out || IW_inst_valid)时,若没有因冲刷而需要丢弃的指令(\textasciitilde(|discard)),则说明马上/现在来的data_ok对应于当前IF的PC,故把指令暂存下来。
\end{enumerate}
% 然后在流入IW时,把inst随流水线流动传递给IW即可,不再赘述。
然后在流入IW时,把inst传递下去即可。

% 然后在流入IW时,把inst随流水线流动传递给IW即可,不再赘述。
% 然后在流入IW时,把inst随流水线流动传递给IW即可,不再赘述。
% 然后在流入IW时,把inst随流水线流动传递给IW即可,不再赘述。

IW阶段保存取回指令的逻辑如下:

\begin{lstlisting}[language=Verilog, caption={IW阶段保存取回指令}]
    always @(posedge clk) begin
        if(rst) begin
            inst_valid <= 1'b0;
            inst <= 32'd0;
        end
        else if(ex_flush || ertn_flush || br_flush) begin
            inst_valid <= 1'b0;
            inst <= 32'd0;
        end
        else if(data_ok && out_ready && (inst_valid_from_IF || inst_valid) && (~(|discard))) begin
            inst_valid <= 1'b1;
            inst <= rdata;
        end
        else if(data_ok && !out_ready && !(inst_valid_from_IF || inst_valid) && (~(|discard))) begin
            inst_valid <= 1'b1;
            inst <= rdata;
        end
        else if(in_valid && ready_go && out_ready) begin
            inst_valid <= 1'b0;
            inst <= 32'd0;
        end
    end
\end{lstlisting}
除了冲刷和指令流动后进行重置以外,可以分成以下两种情况来看:

\begin{enumerate}
    \item IW的PC对应的指令已经收到(inst_valid_from_IF || inst_valid),且这一拍能够流入ID(out_ready):若这时data_ok为1,说明这是当前IF的PC对应的指令,则在下一个上升沿保存下来,同时IF的PC进入IW,正好对应这条指令。
    \item IW的PC对应指令尚未收到(!(inst_valid_from_IF || inst_valid)),且这一拍不能流入ID(!out_ready):若这时data_ok为1,说明这是IW的PC对应的指令,则在下一个上升沿保存下来,等待流入ID。
\end{enumerate}
(注意:上述情况都是在没有需要因冲刷而丢弃指令的情况下(\textasciitilde(|discard))描述的。)

\subsubsection{冲刷流水线时,对异常前已握手成功的取指请求的处理}

我们的设计中,冲刷行为报出时,IF级PC会在同一个周期被设为冲刷后的新值,但此时冲刷前IF可能已经取指请求握手成功,IW可能还未等到冲刷前的请求的data_ok返回——也就是说,接下来会有0到2条应该被冲刷的指令返回,我们需要把它们丢弃掉。discard计数器用于记录还要丢弃的指令数目:

\begin{lstlisting}[language=Verilog, caption={discard计数器更新逻辑}]
    always @(posedge clk) begin
        if(rst) begin
            discard <= 2'd0;
        end
        else if((|discard) && data_ok) begin
            discard <= discard - 2'b01;
        end
        else if(discard_from_IF ^ discard_from_IW) begin
            discard <= discard + 2'b01;
        end
        else if(discard_from_IF && discard_from_IW) begin
            discard <= discard + 2'b10;
        end
    end
\end{lstlisting}

\begin{enumerate}
    \item 若当前还有需要丢弃的指令((|discard)),且收到data_ok,则把discard减1。
    \item 若IF和IW中只有一个阶段报出需要丢弃指令(discard_from_IF ^ discard_from_IW),则把discard加1。
    \item 若IF和IW都报出需要丢弃指令(discard_from_IF \&\& discard_from_IW),则把discard加2。
\end{enumerate}

IF遇到冲刷时,如果之前的请求已经握手成功(handshake_done),且没有返回指令(!inst_valid),则要等这条指令返回后丢弃:

\begin{lstlisting}[language=Verilog, caption={IF阶段-冲刷丢弃指令-控制信号}]
    assign discard_out_wire = (ex_flush || ertn_flush || br_taken) && handshake_done && !inst_valid;
\end{lstlisting}

IW遇到冲刷时,如果冲刷前的PC对应的指令不是从IF传来的(inst_valid_from_IF),也不是之前暂存下来的(inst_valid),并且也没有在报出冲刷的这个周期收到当前PC的data_ok(data_ok \&\& (\textasciitilde(|discard)),其中\textasciitilde(|discard)为1表示这个data_ok不是需要丢弃的,而是恰好对应当前IW的PC),那么这条指令也应该被丢弃:

\begin{lstlisting}[language=Verilog, caption={IW阶段-冲刷丢弃指令-控制信号}]
    wire discard_from_IW = (ex_flush || ertn_flush || br_flush) && in_valid && !(inst_valid_from_IF || (data_ok && (~(|discard))) || inst_valid);
\end{lstlisting}

\subsubsection{MEM级和RDW级访存设计}

% 与IF和IW类似,MEM为了防止重复发取指请求,也引入了handshake_done信号,逻辑与IF类似,不再赘述。对于类SRAM接口新引入的size信号,只需要根据具体访存行为是byte/halfword/word来设置即可,也不再赘述。

与IF和IW类似,MEM为了防止重复发取指请求,也引入了handshake_done信号,逻辑与IF类似,不再赘述。
%%%%%%%%%%%%%%%%%%%%%%%%%%%%%%%%%%%%%%%%%%%%%%%

% 我们在设计wr信号时,结合了写掩码wstrb:写操作时把要改写的字节对应位置1,读操作时全0——这样根据wstrb是否全0就可以设计wr。

% \begin{lstlisting}[language=Verilog, caption={wr信号}]
%     assign wr = (|wstrb);
% \end{lstlisting}

req信号与IF略有不同:

\begin{lstlisting}[language=Verilog, caption={MEM阶段的访存请求IF信号}]
    assign req = in_valid && !handshake_done && !this_flush && (res_from_mem || mem_we);
\end{lstlisting}
在MEM级是有效的访存指令时,若请求没有握手成功(!handshake_done),并且MEM及以后的阶段不存在异常指令(!this_flush),则应当发出访存请求。与IF的区别在于:无论后面的阶段有没有异常/跳转指令,IF都应该持续发取指请求,收到冲刷信号时利用handshake_done_effective信号处理“无效握手”;而MEM及以后的阶段如果存在异常指令,this_flush会被拉高,抑制MEM级访存指令发挥作用,即不允许req拉高、不发请求,所以也就不存在握手成功后收到冲刷信号导致“无效握手”的情况,所以这里也就不必引入handshake_done_effective信号。

也正是因为上述原因,MEM和RDW级不存在“丢弃”操作。

对于load得到的数据的接受与传递,与取指时类似,如果在MEM级收到数据且MEM的指令下一个上升沿无法流入RDW,就在MEM暂存,待RDW允许时流入RDW;RDW也同样设置寄存器保存取回的数据——这方面的设计都与IF/IW类似,故不展示具体代码也不再赘述。
%%%%%%%%%%%%%%%%%%%%%%%%%%%%%%%%%%%%%%%%%%%%%%%%%%%%%%%%%%

\subsubsection{流水控制信号的修改}

引入类SRAM接口并增加了两个流水级后,取指和load/store访存相关流水级的流水控制信号都有所修改(主要是用于表明当前阶段任务完成与否、是否准备好流入下一级的ready_go信号)。

在load/store访存方面:
\begin{enumerate}
    \item MEM级要额外考虑在访存请求握手成功后(当前周期握手成功(req \&\& addr_ok)或是之前已经握手成功(handshake_done)),指令才准备好流入RDW:

    \begin{lstlisting}[language=Verilog, caption={MEM级ready_go信号}]
    assign ready_go = !in_valid  ||
                      this_flush ||
                      !(res_from_mul && !(to_mul_resp_ready && from_mul_resp_valid)) &&
                      !(res_from_div && !(to_div_resp_ready && from_div_resp_valid)) &&
                      !((res_from_mem || mem_we) && !(req && addr_ok || handshake_done));
\end{lstlisting}

    \item RDW级除了考虑无效和冲刷情况,主要应该考虑的是流水级内“没有还未收到data_ok的访存指令”时才算准备好流入WB:
    \begin{lstlisting}[language=Verilog, caption={RDW级ready_go信号}]
        assign ready_go =!in_valid ||
                         this_flush||
                         !((res_from_mem || mem_we) && !(data_valid_from_MEM || data_ok || data_valid));
    \end{lstlisting}
\end{enumerate}

在取指方面,与load/store访存的不同之处在于前面提到的:就算后面流水级有异常/跳转指令,也不会暂停取指,所以与标记异常的this_flush等信号无关。

\begin{enumerate}
    \item IF级只有在当前PC取指请求握手成功(当前周期或者是以前的某个周期)时才算准备好流入IW。特别地,在冲刷报出的那个周期,PC已经修改为冲刷后的值,所以之前就算是有过握手成功且未流入IW的指令,那也是“无效握手”,所以这时IF是否准备好进IW就取决于handshake_done_effective,而非handshake_done:
    \begin{lstlisting}[language=Verilog, caption={IF级ready_go信号}]
        assign ready_go = req && addr_ok || handshake_done_effective;
    \end{lstlisting}

    \item IW级除了无效时一定能流入下一级外,还要考虑:
    \begin{itemize}
        \item 当前收到的指令不是需要丢弃的指令,确保指令与PC一致,才可流入ID;
        \item 当前周期有冲刷信号时,等待指令返回(data_ok)的操作已经交给前面提及的另设的discard相关逻辑进行处理,所以IW流水级相当于“无事可做”,可视为任务完成并准备好流动,故直接拉高ready_go。
    \end{itemize}
    \begin{lstlisting}[language=Verilog, caption={IW级ready_go信号}]
        assign ready_go = !in_valid ||
                           ex_flush || ertn_flush ||
                           br_flush ||
                          (~(|discard)) && (inst_valid_from_IF || data_ok || inst_valid);
    \end{lstlisting}
\end{enumerate}

%%%%%%%%%%%%%%%%%%%%%%%%%%%%%%%%%%%%%%%%%%%%%%%%%%%%outvalid

%%%%%%%%%%%%%%%%%%%%%%%%%%%%%%%%%%%%%%%%%%%%%%%%%%%%%%%%%%%%%%%%%%%%%%%%%%%%%
%%%%%%%%%%%%%%%%%%%%%%%%%%%%%%%%%%%%%%%%%%%%%%%%%%%%%%%%%%%%%%%%%%%%%%%%%%%%%
% \subsubsection{brtaken与brflush与brtakenout}
% for Honglu Ji
%%%%%%%%%%%%%%%%%%%%%%%%%%%%%%%%%%%%%%%%%%%%%%%%%%%%%%%%%%%%%%%%%%%%%%%%%%%%%
%%%%%%%%%%%%%%%%%%%%%%%%%%%%%%%%%%%%%%%%%%%%%%%%%%%%%%%%%%%%%%%%%%%%%%%%%%%%%


\subsection{类 SRAM-AXI 的转接桥设计}

\subsubsection{顶层模块设计}

由于CPU对外只有一个AXI接口,所以转接桥顶层模块要对取指和load/store访存进行仲裁。

我们设立了独热码ar/aw_id,根据优先级表示当前应该与哪个阶段的请求握手(00为无请求,01为取指请求,10为load/store请求),并用于控制状态机的行为。根据教材的提示,load访存请求的优先级大于取指请求,即两者同时存在时,优先与load握手:

\begin{lstlisting}[language=Verilog, caption={CPU读内存请求独热码}]
    assign ar_id = (~sram_wr_2 && sram_req_2) ? 2'b10 : (~sram_wr_1 && sram_req_1) ? 2'b01 : 2'b00;
    assign aw_id = (sram_wr_2 && sram_req_2) ? 2'b10 : 2'b00;
\end{lstlisting}
对于读请求而言,利用ar_id可以选择读请求的size和addr来自IF还是MEM。

特别地,写请求只有store访存会发出,所以aw_id只有00和10两种情况,虽然不用区分对哪一流水级进行握手,但是为了对状态机进行驱动(在00时保持空闲不动,10时才开始工作),仍保留类似ar_id的形式。

此外,为了处理“写后读”相关,引入writing信号,表明当前还有几个store写操作没有处理完:
\begin{lstlisting}[language=Verilog, caption={writing计数器}]
    always @(posedge clk) begin
        if (!resetn) begin
            writing <= 5'b0;
        end
        else if (sram_wr_2 && aw_addr_ok && sram_req_2 && b_data_ok) begin
            writing <= writing;
        end
        else if (sram_wr_2 && aw_addr_ok && sram_req_2) begin
            writing <= writing + 1;
        end
        else if (b_data_ok) begin
            writing <= writing - 1;
        end
    end
\end{lstlisting}

结合以上两方面,就可以对取指和load/store访存的请求进行握手(addr_ok):

\begin{lstlisting}[language=Verilog, caption={顶层模块-CPU读写内存请求-addr_ok(1为取指,2为load/store)}]
    assign sram_addr_ok_1 = ar_id[0] ? ar_addr_ok : 1'b0;
    assign sram_addr_ok_2 = sram_wr_2 ? aw_addr_ok : (writing != 0) ? 1'b0 : ar_id[1] ? ar_addr_ok : 1'b0;
\end{lstlisting}

\begin{enumerate}
    \item aw_addr_ok和aw_addr_ok由状态机发出,表示他们“能够”处理CPU发来的请求。
    \item 但是否向CPU给出addr_ok,还要依赖于这时有无请求(ar_id或aw_id),以及先处理哪一个请求(ar_id,取指或load)。
    \item 此外,对于load/store访存请求,要在写请求全处理完后(writing为0)才能与读请求握手,防止写后读。
\end{enumerate}

与请求类似,对于读数据响应,要根据AXI接口返回的rid信号生成独热码r_id,选择向取指还是load访存进行响应(data_ok),对于写数据响应则不存在选择:

\begin{lstlisting}[language=Verilog, caption={顶层模块-CPU数据响应-data_ok(1为取指,2为load/store)}]
    assign sram_data_ok_1 = r_id[0] && r_data_ok;
    assign sram_data_ok_2 = r_id[1] && r_data_ok || b_data_ok;
\end{lstlisting}

\subsubsection{读请求通道AR}
状态机的转移图如下:

\begin{figure}[H]
    \centering
    \includegraphics[width=0.5\textwidth]{fig/AR状态图.png}
    \caption{读请求通道状态机}
\end{figure}

读请求通道的任务是,响应CPU用类SRAM接口发来的读请求,并通过AXI接口发送给内存。

\begin{enumerate}
    \item 空闲状态IDLE——响应CPU发来的读请求:

    在空闲时,拉高addr_ok,告知顶层模块可以处理读请求(进一步判定交给顶层,上文已经描述):
    \begin{lstlisting}[language=Verilog, caption={AR状态机-addr_ok}]
        assign addr_ok = current_state == IDLE;
    \end{lstlisting}

    由状态图,在收到要马上处理的请求时,IDLE态变为BUSY态,开始处理读请求。

    此外,由于后续通过AXI接口向内存发送读请求时,可能需要持续用到请求信息,但CPU相应流水级收到addr_ok后可能会处理其他指令,导致请求信息丢失,所以读请求通道要用寄存器保存它们:

    \begin{lstlisting}[language=Verilog, caption={AR暂存请求信息}]
        always @(posedge clk) begin
        if (!resetn) begin
            id_reg <= 2'b0;
            addr_reg <= 32'b0;
            size_reg <= 2'b0;
        end
        else if (current_state == IDLE) begin
            id_reg <= id;
            addr_reg <= addr;
            size_reg <= size;
        end
    end
    \end{lstlisting}

    \item BUSY状态——通过AXI接口把收到的读请求发给内存:

    这时只需拉高arvalid信号,并把保存下来的请求信息通过AXI接口发送:

    \begin{lstlisting}[language=Verilog, caption={AXI接口读通道信号}]
        assign arid = {3'b0, id_reg[1]};
        assign araddr = addr_reg;
        assign arsize = {1'b0, size_reg};
        assign arvalid = current_state == BUSY;
    \end{lstlisting}

    由状态图,当用AXI接口发送的请求与内存握手成功后(arready \&\& arvalid),本次任务完成,回到IDLE,等待CPU发来的下一个读请求。
\end{enumerate}


\subsubsection{读响应通道R}

此通道的作用是接收内存通过AXI接口返回的读数据相关信号,然后通过类SRAM接口传递给CPU。

由于类SRAM接口的读数据没有握手,直接返回数据和data_ok即可,所以面向AXI接口一侧的rready可以在解除复位后一直拉高:

\begin{lstlisting}[language=Verilog, caption={rready信号}]
    assign rready = resetn;
\end{lstlisting}

要做的工作十分简单,除了接受数据,再把rid转换成顶层模块使用的独热码(id),并把握手成功的信号转化成data_ok即可:
\begin{lstlisting}[language=Verilog, caption={读响应通道工作}]
    always @(posedge clk) begin
        if (!resetn) begin
            id <= 2'b0;
            data_ok <= 1'b0;
            data <= 32'b0;
        end
        else begin
            id <= {rid[0], ~rid[0]};
            data_ok <= rvalid && rready;
            data <= rdata;
        end
    end
\end{lstlisting}


\subsubsection{写请求AW与写数据W通道}

写请求与写数据利用同一个状态机实现,状态转移图如下:

\begin{figure}[H]
    \centering
    \includegraphics[width=0.5\textwidth]{fig/AW状态图.png}
    \caption{写请求与写数据通道状态机}
\end{figure}

这一模块的任务是,响应CPU发的写请求和数据,并把它通过AXI接口发送给内存。

\begin{enumerate}
    \item 初始化后,处于IDLE。根据顶层模块发来的写请求独热码(id),检测到写请求后便启动,转到BUSY态。
    \item 在BUSY态,同时拉高awvalid和wvalid信号,通过AXI接口发送写请求和写数据。
    \begin{enumerate}
        \item 若二者同时握手成功,则本次任务完成,回到IDLE,等待下一个写请求和数据。
        \item 若aw通道握手成功,则跳转到AW_FIRE态,不再拉高awvalid(防止重复发请求),但保持wvalid拉高,继续等待w通道握手成功,成功后回到IDLE。
        \item 类似地,若w通道握手成功,则跳转到W_FIRE态,不再拉高wvalid(防止重复发请求),但保持awvalid拉高,继续等待aw通道握手成功,成功后回到IDLE。
    \end{enumerate}
\end{enumerate}
结合上面的阐释,awvalid和wvalid信号定义如下:

\begin{lstlisting}[language=Verilog, caption={awvalid和wvalid信号}]
    assign awvalid = current_state == BUSY || current_state == W_FIRE;
    assign wvalid = current_state == BUSY || current_state == AW_FIRE;
\end{lstlisting}

与读请求状态机类似,本状态机也要把发给AXI接口的请求信息保存于寄存器中。

%%%%%%%%%%%%%%%%%%%%%%%%%%%%%%%%%%%%%%%%%%%addr_ok

\subsubsection{写响应通道B}

与读响应通道类似,在解除复位后拉高bready,一直能够接收内存从AXI接口发回的响应信号。data_ok的设置类似读响应通道,不再赘述。

\section{Debug记录}

\subsection{转接桥写请求与写数据状态机逻辑错误}

按照教材要求,我们把写请求和写数据通道写在一个状态机中,想当然地认为写请求和写数据会同时得到ready信号,于是只设计了IDLE和BUSY态,在BUSY态完成握手后回到IDLE,但仿真时出现了下图错误:

\begin{figure}[H]
    \centering
    \includegraphics[width=0.9\textwidth]{fig/总线桥读请求与读数据通道状态机逻辑错误.png}
    \caption{转接桥写请求与写数据状态机逻辑错误-导致请求重复发送-波形}
\end{figure}
在光标位置附近的几个周期内,状态机处于BUSY,持续拉高awvalid和wvalid信号,但awready在光标左侧位置就已经发来,wready在光标右侧位置才发来。在这一版设计中,由于没有完成两个通道的握手,所以状态机一直处于BUSY,一直在发请求。在光标右侧的那个周期依旧拉高的awvalid又被内存接收到一次,导致下一个周期又回来一次awready——于是,同一条指令的内存写请求做了两次握手,写数据做了一次握手,最终导致后面继续运行时出现卡死。

所以我们后来增加了这一状态机的状态数,如果只完成了一个通道的握手,就转到W_FIRE或AW_FIRE,拉低已经握手成功的valid信号,保留没成功的那个通道的valid,等其握手成功后才能回到IDLE。

\subsection{writing信号提前拉低导致写后读处理错误}

最初我们在做设计时,只考虑了在收到store写请求后,拉高writing,写响应发回后,拉低writing的情况,然后在收到load读请求时如果writing为1就不与CPU握手。然而,仿真时遇到如下问题:

\begin{figure}[H]
    \centering
    \includegraphics[width=0.9\textwidth]{fig/writing.png}
    \caption{writing信号提前拉低导致写后读处理错误-波形}
\end{figure}

如上图所示,图中有连续的三条访存指令——store-store-load,在反汇编文件中是下面这部分:

\begin{figure}[H]
    \centering
    \includegraphics[width=0.6\textwidth]{fig/ssl.png}
    \caption{store-store-load访存指令序列}
\end{figure}

在波形图的红框部分,两条store指令请求握手成功,在发第一条请求时writing被拉高,行为正确;黄框和白框部分分别收到两条指令的写响应。然而,第一次响应时writing就被拉低了,其无法正确反应当前还有一条握手成功写请求没有得到响应;进一步地,由于writing拉低,蓝框部分的load读请求向内存发了出去,导致写后读相关变成了先读后写,最终引发错误。

修改方法上文已经提及,把writing由标志位改为计数器,每多一个store写请求就加1,收到响应就减1,减到0时才能允许load读请求发给内存。

\subsection{this_flush信号设计错误}

在上一个project中,由于访存没有延迟,所以出现异常的指令与前面流水级的指令之间不会有空泡,于是我们设计的this_flush信号如下(这一信号用于标记当前流水级的指令是否在接下来的某一刻被冲刷,阻止这些会被冲刷的指令改变体系状态):

\begin{lstlisting}[language=Verilog, caption={上一个Project的this_flush信号(以ID为例)}]
    this_flush = in_valid && (has_exception || next_flush || SYSCALL || BRK || INE || INT || inst_ertn);
\end{lstlisting}
需要考虑当前级指令在流经前面流水级时有没有异常(has_exception),当前流水级有没有异常(SYSCALL等信号),以及下一流水级的指令有没有异常(next_flush)——由于没有空泡,这是合理的。


然而,在本Project中,访存具有延迟,就可能出现两个有效流水级之间存在无效流水级(空泡),无效流水级自然也不会有this_flush。这时只用next_flush就会出错,比如:MEM级存在异常指令,EX级是空泡,ID级的指令之后会被冲刷(this_flush应为1),但MEM的this_flush无法通过next_flush链条经EX传到ID,导致ID的this_flush实际不为1,遂出错。

修改后,我们把每一级this_flush逻辑中的next_flush改为了其后所有流水级的this_flush之“或”:

\begin{lstlisting}[language=Verilog, caption={本Project修改后的this_flush信号(以ID为例)}]
    assign this_flush = in_valid && (has_exception || EX_flush || MEM_flush || RDW_flush || WB_flush || SYSCALL || BRK || INE || INT || inst_ertn);
\end{lstlisting}

\subsection{冲刷时指令丢弃数目统计错误}

我们在CPU设计时犯了一个与上面的writing信号类似的错误:

\begin{figure}[H]
    \centering
    \includegraphics[width=0.9\textwidth]{fig/指令丢弃数目错误.png}
    \caption{指令丢弃数目错误-波形}
\end{figure}

如上图所示,红框部分出现ertn冲刷,IF和IW都判断出自身流水级内部还有冲刷前请求成功但没收到的指令(需要扔掉),两个discard_from都拉高,即要丢掉两条指令,但(蓝框所示的)discard最初实现为标志位,导致第一个data_ok到来时(黄框所示)就被拉低。从而使得后面第二个(要扔掉的)data_ok到来时因discard为0而被误视为是当前IW阶段的PC对应的正确指令,从而出现指令和PC错位,最终引发错误。

修改方法类似writing,改为计数器即可。

\subsection{discard_from_IW信号逻辑不完备}

在最初的设计中,discard_from_IW信号设计如下:

\begin{lstlisting}[language=Verilog, caption={最初的discard_from_IW}]
    wire discard_from_IW = (ex_flush || ertn_flush || br_flush) && in_valid && !(inst_valid_from_IF || data_ok || inst_valid);
\end{lstlisting}

然后出现了下图所示的错误波形:

\begin{figure}[H]
    \centering
    \includegraphics[width=0.9\textwidth]{fig/discard_IW.png}
    \caption{discard_from_IW信号逻辑不完备导致的错误-波形}
\end{figure}

在红框所示位置,报出跳转冲刷br_flush,由于IW当前PC还未收到data_ok,故拉高discard_from_IW,又由于discard_from_IF为0,故上升沿到来时discard计数器从0更新为1(如蓝框所示)。在上升沿到来后的这一周期,WB报出ertn冲刷,同时收到一个data_ok(白框所示),这时我们的设计会误认为data_ok是当前IW的PC对应的指令(但由于discard为1,实则并不对应),相当于判定为IW阶段没有没收到的data_ok,故黄框位置处discard_from_IW为0(应该为1),导致少丢弃一条指令,出现后续指令与PC对应不上的错误。

于是,我们把计数器discard纳入了discard_from_IW的逻辑以避免上述误判:

\begin{lstlisting}[language=Verilog, caption={修改后的discard_from_IW}]
    wire discard_from_IW = (ex_flush || ertn_flush || br_flush) && in_valid && !(inst_valid_from_IF || (data_ok && (~(|discard))) || inst_valid);
\end{lstlisting}


\section{合作说明}

本实验由本组成员共同合作完成,组内同学同等贡献。



\end{document}