% 这是中国科学院大学计算机科学与技术专业《计算机组成原理(研讨课)》使用的实验报告 Latex 模板
% 本模板与 2024 年 2 月 Jun-xiong Ji 完成, 更改自由 Shing-Ho Lin 和 Jun-Xiong Ji 于 2022 年 9 月共同完成的基础物理实验模板
% 如有任何问题, 请联系: jijunxoing21@mails.ucas.ac.cn
% This is the LaTeX template for report of Experiment of Computer Organization and Design courses, based on its provided Word template.
% This template is completed on Febrary 2024, based on the joint collabration of Shing-Ho Lin and Junxiong Ji in September 2022.
% Adding numerous pictures and equations leads to unsatisfying experience in Word. Therefore LaTeX is better.
% Feel free to contact me via: jijunxoing21@mails.ucas.ac.cn

\documentclass[11pt]{article}

\usepackage[a4paper]{geometry}
\geometry{left=2.0cm,right=2.0cm,top=2.5cm,bottom=2.5cm}

\usepackage{ctex} % 支持中文的LaTeX宏包
\usepackage{amsmath,amsfonts,graphicx,subfigure,amssymb,bm,amsthm,mathrsfs,mathtools,breqn} % 数学公式和符号的宏包集合
\usepackage{algorithm,algorithmicx} % 算法和伪代码
\usepackage[noend]{algpseudocode} % 算法和伪代码
\usepackage{fancyhdr} % 自定义页眉页脚
\usepackage[framemethod=TikZ]{mdframed} % 创建带边框的框架
\usepackage{fontspec} % 字体设置
\usepackage{adjustbox} % 调整盒子大小
\usepackage{fontsize} % 设置字体大小
\usepackage{tikz,xcolor} % 绘制图形和使用颜色
\usepackage{multicol} % 多栏排版
\usepackage{multirow} % 表格中合并单元格
\usepackage{pdfpages} % 插入PDF文件
\usepackage{listings} % 在文档中插入源代码
\usepackage{wrapfig} % 文字绕排图片
\usepackage{bigstrut,multirow,rotating} % 支持在表格中使用特殊命令
\usepackage{booktabs} % 创建美观的表格
\usepackage{circuitikz} % 绘制电路图
\usepackage{zhnumber} % 中文序号(用于标题)
\usepackage{tabularx} % 表格折行

\definecolor{dkgreen}{rgb}{0,0.6,0}
\definecolor{gray}{rgb}{0.5,0.5,0.5}
\definecolor{mauve}{rgb}{0.58,0,0.82}
% \lstset{
%   frame=tb,
%   aboveskip=3mm,
%   belowskip=3mm,
%   showstringspaces=false,
%   columns=flexible,
%   framerule=1pt,
%   rulecolor=\color{gray!35},
%   backgroundcolor=\color{gray!5},
%   basicstyle={\small\ttfamily},
%   numbers=none,
%   numberstyle=\tiny\color{gray},
%   keywordstyle=\color{blue},
%   commentstyle=\color{dkgreen},
%   stringstyle=\color{mauve},
%   breaklines=true,
%   breakatwhitespace=true,
%   tabsize=3,
% }



\lstset{
    % basicstyle=\ttfamily\small,
    basicstyle=\ttfamily\footnotesize,
    keywordstyle=\color{blue},
    commentstyle=\color{gray},
    numbers=left,
    numberstyle=\tiny,
    breaklines=true,
    frame=single,
}

\lstdefinelanguage{Verilog}{
    morekeywords={always, and, assign, attribute, begin, buf, bufif0, bufif1, case, casex, casez,
                 cmos, deassign, default, defparam, disable, edge, else, end, endattribute,
                 endcase, endfunction, endmodule, endprimitive, endspecify, endtable, endtask,
                 event, for, force, forever, fork, function, highz0, highz1, if, ifnone, initial,
                 inout, input, integer, join, large, macromodule, medium, module, nand, negedge,
                 nmos, nor, not, notif0, notif1, or, output, parameter, pmos, posedge, primitive,
                 pull0, pull1, pulldown, pullup, rcmos, real, realtime, reg, release, repeat,
                 rnmos, rpmos, rtran, rtranif0, rtranif1, scalared, signed, small, specify,
                 specparam, strong0, strong1, supply0, supply1, table, task, time, tran, tranif0,
                 tranif1, tri, tri0, tri1, triand, trior, trireg, unsigned, vectored, wait, wand,
                 weak0, weak1, while, wire, wor, xnor, xor},
    morekeywords=[2]{`define, `undef, `ifdef, `ifndef, `else, `elsif, `endif, `include,
                    `resetall, `timescale, `celldefine, `endcelldefine, `default_nettype,
                    `unconnected_drive, `nounconnected_drive},
    morekeywords=[3]{\$display, \$write, \$strobe, \$monitor, \$time, \$stime, \$realtime,
                    \$finish, \$stop, \$dumpfile, \$dumpvars, \$dumpon, \$dumpoff, \$dumpall,
                    \$dumpflush, \$dumplimit, \$random, \$readmemb, \$readmemh},
    morecomment=[l]{//},
    morecomment=[s]{/*}{*/},
    morestring=[b]",
    sensitive=true,
    keywordstyle=\color{blue}\bfseries,
    keywordstyle=[2]\color{purple}\bfseries,
    keywordstyle=[3]\color{orange}\bfseries,
    stringstyle=\color{red},
    numberstyle=\tiny\color{gray},
    stepnumber=1,
    numbersep=10pt,
    backgroundcolor=\color{white},
    showspaces=false,
    showstringspaces=false,
    showtabs=false,
    frame=single,
    tabsize=4,
    captionpos=b,
    breaklines=true,
    breakatwhitespace=false,
    title=\lstname,
    escapeinside={\%*}{*},
    morekeywords={*,...}
}









% 轻松引用, 可以用\cref{}指令直接引用, 自动加前缀.
% 例: 图片label为fig:1
% \cref{fig:1} => Figure.1
% \ref{fig:1}  => 1
\usepackage[capitalize]{cleveref}
% \crefname{section}{Sec.}{Secs.}
\Crefname{section}{Section}{Sections}
\Crefname{table}{Table}{Tables}
\crefname{table}{Table.}{Tabs.}

% \setmainfont{Palatino Linotype.ttf}
% \setCJKmainfont{SimHei.ttf}
% \setCJKsansfont{Songti.ttf}
% \setCJKmonofont{SimSun.ttf}
\punctstyle{kaiming}
% 偏好的几个字体, 可以根据需要自行加入字体ttf文件并调用

\renewcommand{\emph}[1]{\begin{kaishu}#1\end{kaishu}}

% 对 section 等环境的序号使用中文
% \renewcommand \thesection{\zhnum{section}、}
% \renewcommand \thesubsection{\arabic{section}}


%%%%%%%%%%%%%%%%%%%%%%%%%%%
%改这里可以修改实验报告表头的信息
\newcommand{\name}{裴晨皓\ 竹彦博\ 纪弘璐}
\newcommand{\studentNum}{2023K8009916003\ \ \ \ 2023K8009916001\ \ \ \ 2023K8009916002}
\newcommand{\major}{计算机科学与技术}
\newcommand{\labNum}{Project 6}
\newcommand{\labName}{TLB MMU设计专题实验}
%%%%%%%%%%%%%%%%%%%%%%%%%%%

\begin{document}

\input{tex_file/head.tex}

\section{实验简介}

这次实验中,本组同学按照讲义提示,先实现了独立的TLB模块,又实现了用于虚实地址转换的MMU模块,并将新增模块与已有的流水线和CSR模块进行协同,又依次完成了TLB指令的添加和TLB异常的处理,同时在设计过程中兼顾了CSR寄存器的添加及其读写逻辑的设计。

\section{设计方案介绍}
下图是完成本次实验后的处理器结构框图:

\begin{figure}[H]
    \centering
    \includegraphics[width=0.84\textwidth]{fig/结构框图.png}
    \caption{处理器结构框图}
\end{figure}

\subsection{总体思路}

我们在本次实验新增加了MMU和TLB模块,上面的结构框图还不够直观,于是我们绘制了下面的逻辑示意图,直观展示新增的两个模块与流水线和CSR模块实现协同工作的逻辑:

\begin{figure}[H]
    \centering
    \includegraphics[width=0.58\textwidth]{fig/协同.png}
    \caption{新增模块与原有模块协调工作逻辑图}
\end{figure}

\subsection{TLB模块设计}

\subsubsection{组织结构与接口}
TLB的主体是一个二维组织结构的查找表,每一项都有多个字段,所以类似寄存器堆的实现,内部要用reg对信息进行保存;由于每个字段的含义不同,所以仿照CSR模块的写法,不把每一项当成整体实现,而是把不同字段用reg分别实现。

对外的接口具体为:
\begin{enumerate}
    \item 两套查找接口:取指和Load/Store访存操作都涉及查找TLB,为了保证流水线的性能良好,它们需要能够同时对TLB进行查找,故它们各自使用一套查找接口;由于INVTLB和TLBSRCH指令也都需要查找,但它们和Load/Store访存指令一样,都在MEM阶段进行查找,故复用Load/Store访存使用的查找接口(然后只需要附加它们额外需要的invtlb_op 输入和s_index输出即可)。
    \item 读写接口各一套:用于支持专门的根据index对TLB进行读(TLBRD)写(TLBWR 和 TLBFILL)的指令。
\end{enumerate}


\subsubsection{读写功能}

读写功能与寄存器堆的读写类似,都是同步写异步读。需要注意的是要把内部保存的1位ps字段和读写时输入输出的6位ps进行转换(查找时也需要转换):
\begin{lstlisting}[language=Verilog, caption={ps字段的处理}]
    assign r_ps   = tlb_ps4MB[r_index] ? 6'd21 : 6'd12;
    always @(posedge clk) begin
        if(we) begin
                tlb_ps4MB[w_index] <= (w_ps == 6'd21);
        end
    end
\end{lstlisting}
ps为21对应4MB页,ps为12对应4KB页,模块内每一项的tlb_ps4MB信号为1时对应4MB页,否则对应4KB页。


\subsubsection{查找功能}

关于查找功能,按照讲义的提示,1号接口(Load/Store访存指令在MEM级使用)可以与INVTLB指令op为6的情况复用逻辑。于是,我们先把INVTLB用来查找、判断的condition进行实现,然后再用它们生成正常查找和INVTLB时的查找结果:

\begin{lstlisting}[language=Verilog, caption={正常查找与INVTLB查找逻辑}]
for(i = 0; i < TLBNUM; i = i + 1)begin
    assign    cond1   [i] =  (tlb_g[i] == 1'b0);
    assign    cond2   [i] =  (tlb_g[i] == 1'b1);
    assign    s1_cond3[i] =  (s1_asid == tlb_asid[i]);
    assign    s0_cond3[i] =  (s0_asid == tlb_asid[i]);
    assign    s1_cond4[i] =  (s1_vppn[18:9] == tlb_vppn[i][18:9]) 
                          && (tlb_ps4MB[i] || (s1_vppn[8:0] == tlb_vppn[i][8:0]));
    assign    s0_cond4[i] =  (s0_vppn[18:9] == tlb_vppn[i][18:9]) 
                          && (tlb_ps4MB[i] || (s0_vppn[8:0] == tlb_vppn[i][8:0]));

    assign match1[i] = tlb_e[i] && s1_cond4[i] && (cond2[i] || s1_cond3[i]);
    assign match0[i] = tlb_e[i] && s0_cond4[i] && (cond2[i] || s0_cond3[i]);

    assign inv_match[i] =  (invtlb_op == 5'h0 || invtlb_op == 5'h1) && (cond1[i] || cond2[i])
                        || (invtlb_op == 5'h2) && cond2[i]
                        || (invtlb_op == 5'h3) && cond1[i]
                        || (invtlb_op == 5'h4) && cond1[i] && s1_cond3[i] 
                        || (invtlb_op == 5'h5) && cond1[i] && s1_cond3[i] && s1_cond4[i]
                        || (invtlb_op == 5'h6) && match1[i];
end
\end{lstlisting}

\begin{itemize}
    \item cond1和cond2只与TLB内部的信息有关,故各自只有1个信号;而cond3和cond4还额外涉及两个查找接口的外部输入,故它们各自应当有2个信号,分别对应两个接口(s0、s1)。
    \item 正常查找时(除了INVTLB),命中第i项对应于match0/1[i]拉高。虽然与INVTLB在op为6时逻辑类似,但并不完全相同——这里要额外考虑tlb_e,因为这一项如果无效,就不能算命中,也就不能把match拉高。
    \item 对于INVTLB用到的查找,命中第i项对应于inv_match[i]拉高。由于该指令会把所有符合条件的项的e位置0,所以inv_match的逻辑中可以不加入tlb_e,因为本来无效的一项再清0一次也无妨。op为0至5时根据指令集手册用cond逻辑进行设计,op为6时复用MEM级Load/Store访存用到的1号查找接口的match1信号。
\end{itemize}

若是一般查找,那么根据上面提到的match信息,就可以判断本次查找是否命中——若某一项的match拉高,则说明命中。故我们可以将match向量的所有位“或”在一起进行判断:
\begin{lstlisting}[language=Verilog, caption={命中判断信号——found(以取指时用到的0号查找接口为例)}]
assign s0_found = |match0;
\end{lstlisting}
具体命中哪一项(s0/1_index),根据match向量的所有位实现一个多路选择器来判断即可,不再赘述。

在查找时还要注意TLB里“一项对应两页”的特性——确定了命中项后,还要对其中的两页进行“二选一”:

\begin{lstlisting}[language=Verilog, caption={对一项中的两页进行选择(以0号查找接口为例)}]
assign s0_select_unit1 = tlb_ps4MB[s0_index] && s0_vppn[8] || !tlb_ps4MB[s0_index] && s0_va_bit12;
\end{lstlisting}

设同一项中的两页里,虚页号最低位(第0位)为0的一页为第0页,虚页号最低位为1的一页为第1页,则s0_select_unit1信号拉高时表示选中第1页。此外,在我们的设计中,vppn信号的第0位对应的是4KB页的虚双页号的第0位,而非4MB页的虚双页号的第0位,这一点要注意。

关于s0_select_unit1信号,下面是两个易错的细节,理解不到位或是数位数时粗心就可能出错:
\begin{itemize}
    \item 4KB页:
    
    这时,页内偏移为虚地址的第0到11位,虚页号从虚地址第12位开始算;虚页号最低位不放入TLB,虚双页号从第13位开始算。所以这里要根据s0_va_bit12来选择一项中的两页,而非vppn中的某一位。

    \item 4MB页:
    
    根据指令集手册,每一项里的两页实际上都是2MB大小。由于$2MB = 2^{21}B$,所以虚页号的第0位应该看虚地址的第21位,也就是虚双页号的第8位。
\end{itemize}

特别地,若当前是INVTLB(invtlb_valid拉高)指令用到的查找,直接把命中项的e位置0即可,不涉及项内页选择:

\begin{lstlisting}[language=Verilog, caption={INVTLB对命中项的e位置0逻辑}]
always @(posedge clk) begin
        if(we)
                tlb_e[w_index] <= w_e;
        else if(invtlb_valid) begin      
                for(j = 0; j < TLBNUM; j = j + 1) begin
                        if(inv_match[j])
                                tlb_e[j] <= 1'b0;
                end
        end
end
\end{lstlisting}

\subsection{MMU模块设计}

如图2所示,我们的MMU模块主要负责接收流水线发来的虚地址,进行虚实地址转换操作,并将转换后的物理地址送往AXI转接桥。如果转换过程涉及TLB查找,则还要对TLB相关异常进行判定并发回流水线。

\subsubsection{地址翻译逻辑}

以取指为例:

\begin{lstlisting}[language=Verilog, caption={虚实地址翻译-取指}]
assign inst_sram_paddr = 
(crmd_da_value && !crmd_pg_value) ? 
    inst_sram_vaddr : 
(inst_sram_vaddr[31: 29] == dmw0_vseg_value && dmw0_plv_cond) ? 
    {dmw0_pseg_value, inst_sram_vaddr[28: 0]} :
(inst_sram_vaddr[31: 29] == dmw1_vseg_value && dmw1_plv_cond) ? 
    {dmw1_pseg_value, inst_sram_vaddr[28: 0]} :
{tlb_s0_ppn, inst_sram_vaddr[11: 0]};
\end{lstlisting}
\begin{itemize}
    \item 若DA为1且PG为0,直接把虚地址作为物理地址输出。
    \item 否则再看两个直接映射窗口,若权限等级符合(dmw0/1_plv_cond为1),并且虚地址的高3位与对应窗口的dmw0/1_vseg_value相等,则直接用dmw0/1_pseg_value与虚地址的低29位拼接得到物理地址。
    \item 否则,使用TLB查找得到的ppn与虚地址的低12位(页内偏移)拼接得到物理地址。
\end{itemize}

关于使用直接映射窗口时的权限等级判定(dmw0/1_plv_cond),只需根据CRMD寄存器的PLV值与DMW0/1寄存器记录的PLV情况进行比较即可,二者相同说明权限级别符合,dmw0/1_plv_cond拉高。此处不再赘述。

\subsubsection{TLB异常判定}

TLB异常发生在借助TLB进行地址转换时,所以发生此类异常的必要条件是地址转换“的确使用了TLB”,同样以取指为例:

\begin{lstlisting}[language=Verilog, caption={对地址翻译是否使用了TLB的判定}]
wire use_tlb_i = !(crmd_da_value && !crmd_pg_value) &&
        !(inst_sram_vaddr[31: 29] == dmw0_vseg_value && dmw0_plv_cond) &&
        !(inst_sram_vaddr[31: 29] == dmw1_vseg_value && dmw1_plv_cond);
\end{lstlisting}
即:既不是“DA为1且PG为0”的直接地址翻译,也不是落在两个直接映射窗口内的情况,那就一定用到了TLB。进一步,再结合查找TLB得到的信息,就可以产生异常标志信号:
\begin{lstlisting}[language=Verilog, caption={异常标志-以取指为例}]
wire tlbr_i = use_tlb_i && !tlb_s0_found;
wire pif_i = use_tlb_i && tlb_s0_found && !tlb_s0_v;
wire ppi_i = use_tlb_i && tlb_s0_found && tlb_s0_v && (crmd_plv_value > tlb_s0_plv);
\end{lstlisting}
这里要注意的是,取指时仅会出现TLB重填异常、取指页无效异常和页特权等级不合规异常(共3类);而访存级还会出现load/store页无效异常和页修改异常,但不会出现取指页无效异常(共5类)。

最后,还要根据上面的异常标志,把异常信息通过ecode和esubcode的形式发回流水线,此处仍以取指的ecode为例:
\begin{lstlisting}[language=Verilog, caption={ecode生成逻辑-取指}]
assign ecode_i = tlbr_i ? 6'h3F :
                 pif_i  ? 6'h3 :
                 ppi_i  ? 6'h7 :
                          6'h0;
\end{lstlisting}
% % % % % % (如果没有异常发生,则ecode为0)。


\subsection{TLB指令的实现与TLB/CSR冲突处理}

在我们的设计中,TLB指令对TLB的读、写和查找都在MEM级进行。下面介绍实现过程中的关键:

\subsubsection{重取机制}
根据讲义的提示,由于实现的几条TLB指令(除了TLBSRCH),都会修改TLB的内容或是虚实地址转换所依赖的CSR的内容,而这些内容最早在取指级做地址翻译就会被使用,所以这些TLB指令后面几条已经取出的/发过取指请求的指令,很可能用的是旧的TLB/CSR信息进行的地址翻译,所以需要进行“重取”。

此外,CSR指令也有可能改写地址翻译依赖的CRMD、ASID、DMW0或DMW1寄存器,那么后续指令也要重取。

借鉴异常冲刷的处理逻辑,仿照this_flush信号(用于标记当前流水级的指令将来会被冲刷,以此限制指令做出修改内存等改变体系状态的操作),我们引入this_tlb_refetch/flush和this_csr_refetch/flush信号,限制因重取而被冲刷的指令的行为。

% \begin{lstlisting}[language=Verilog, caption={this_tlb_refetch和this_csr_flush信号-以ID为例}]
% assign this_tlb_refetch = 
%             in_valid 
%         && (inst_tlbsrch || inst_tlbrd || inst_tlbwr || inst_tlbfill || inst_invtlb 
%             || EX_this_tlb_refetch || MEM_this_tlb_refetch || RDW_this_tlb_refetch);
% assign this_csr_flush = in_valid && EX_this_csr_flush;
% \end{lstlisting}
% 由于无阻塞情况下,TLB指令对TLB的写入在从MEM进入RDW的上升沿才能完成,所以我们认为其在RDW级“提交”重取;同理CSR指令需要重取时,在EX级“提交”重取——尽早地“提交”有利于提高性能。
% \begin{itemize}
%     \item this_tlb_refetch:
    
%     本级有效,并且本级是TLB指令或是后续直到RDW阶段有因TLB指令而重取的情况,就给本级标上需要重取;
%     \item this_csr_flush:
    
%     本级有效,并且后续EX存在因CSR指令而重取的情况,就给本级标上需要重取。
    
%     这里不用“或”上“本级是影响地址翻译的CSR写指令”,是因为要重取的是“影响地址翻译的CSR写指令”之后的指令,不包括这条指令本身。同理,TLB指令带来的重取也是面向其后的指令,不包括其本身,所以在MEM级的this_tlb_refetch也有类似处理,不再赘述。
% \end{itemize}

以MEM级的this_tlb_refetch和this_tlb_flush信号为例:
\begin{lstlisting}[language=Verilog, caption={this_tlb_refetch和this_tlb_flush信号-以MEM为例}]
assign this_tlb_refetch = in_valid 
                        && (tlbsrch || tlbrd || tlbwr || tlbfill || invtlb 
                           || RDW_this_tlb_refetch);
assign this_tlb_flush = in_valid && RDW_this_tlb_refetch;
\end{lstlisting}
由于无阻塞情况下,TLB指令对TLB的写入在从MEM进入RDW的上升沿才能完成,所以我们认为其在RDW级“提交”重取;同理CSR指令导致重取时,在EX级“提交”重取——尽早地“提交”有利于提高性能。
\begin{itemize}
    \item this_tlb_refetch:表明本级的指令与重取相关。
    
    本级有效,并且本级是TLB指令或是后续直到(做重取提交的)RDW的所有阶段中,存在与重取有关的情况,那么本级指令也一定与重取有关。
    \item this_csr_flush:表明本级的指令将来会因重取而被冲刷。
    
    本级有效,并且后续直到RDW的所有阶段中,存在有关于重取的情况,说明本级的指令将来一定会被重取。
\end{itemize}
refetch和flush的区别在于,前者包含了“本级是引起重取的指令”的情况,而后者不包含。为了防止引起重取的指令作用也被限制,所以在其发挥作用的MEM级区分出这两个信号(用flush限制指令的功能而非用refetch),而前面的流水级中,引起重取的指令本来也不会产生作用,其作用是否受限不会带来影响,故只使用一个refetch信号即可。


重取被提交后,前面各流水级复用异常时的冲刷逻辑进行冲刷即可,冲刷后的IF级取指PC应当设为被重取的第一条指令(即引起重取的指令的下一条指令)的PC。




\subsubsection{“冲突”处理}

由于TLB指令对TLB的读写都在MEM,因此不会出现围绕TLB的读写相关冲突,只需要做好前面提到的重取即可。

然而,原有的设计中,CSR指令对CSR的读写都在ID进行;现在(除了ID)IF和MEM都会读CSR,在MEM更是有可能出现写CSR的TLBSRCH、TLBRD指令。由于读与读之间不互斥,而读与写、写与写是互斥的,所以我们重点考虑涉及写的冲突处理。

\noindent
(特别地,WB级的异常提交与返回也涉及CSR的写,但这种情况下前面的指令会被冲刷,将会被冲刷的指令不会产生作用,相当于不读也不写CSR,所以没有冲突。)

\begin{enumerate}
    

    \item MEM写,ID写:
    
    此时MEM为TLB指令,前面的流水级的refetch信号都会拉高,阻止前面即将被重取冲刷的指令发挥作用,故实际上ID不会出现“写”CSR的行为,所以不必担心:

    \begin{lstlisting}[language=Verilog, caption={ID级refetch信号对csr写使能的限制}]
    assign csr_we = in_valid && (inst_csrwr || inst_csrxchg) && ready_go && out_ready 
                  && !this_flush && !this_tlb_refetch && !this_csr_flush;
    \end{lstlisting}
    
    \item MEM写,ID读:
    
    与上一条类似,ID这时被refetch限制,ID这条指令将来会被冲刷,所以这时CSR的读端口读出什么值也就不必在意了。

    \item MEM读,ID写:
    
    \begin{enumerate}
        \item MEM为TLB指令,ID为CSR指令:
        
        同样,ID的指令被refetch限制,不必在意二者冲突。

        \item MEM为正常访存指令,ID为CSR指令:
        
        这时没有refetch对ID进行限制(不需要重取),若ID的写CSR操作发生在MEM读CSR之前,则可能使MEM读出错误的值。于是,要对ID进行阻塞(当前设计中CSR写操作在CSR写指令从ID进入EX的上升沿发生,所以只要写指令阻塞在ID,就不会对CSR进行写入)。

        \begin{lstlisting}[language=Verilog, caption={csr_affect_mem信号的生成}]
wire csr_affect_mem = in_valid && (inst_csrwr || inst_csrxchg) 
                    && (  inst[23: 10] == `CSR_CRMD || inst[23: 10] == `CSR_ASID 
                       || inst[23: 10] == `CSR_DMW0 || inst[23: 10] == `CSR_DMW1);
        \end{lstlisting}
        首先,要判定ID的CSR指令是否会改写地址翻译所依赖的CSR(CRMD、ASID、DMW0、DMW1),得到上面的csr_affect_mem信号。若它不改写这些CSR,则不会影响MEM的地址翻译,为了保障性能,可以不阻塞。

        然后,再根据EX和MEM的指令类型(是否为load/store),生成ID级的阻塞信号:
        \begin{lstlisting}[language=Verilog, caption={CSR读写相关的阻塞信号-ID级}]
assign csr_mem_stall = in_valid && (EX_mem_inst || MEM_mem_inst) && csr_affect_mem;
        \end{lstlisting}
    \end{enumerate}
\end{enumerate}


\subsubsection{指令功能实现}
我们把流水线中关于TLB指令的信号都送入CSR,由CSR模块把这些指令的信号送入TLB。如果它们涉及CSR的读写,能够较为方便地在CSR模块内进行处理。

对于TLBSRCH和INVTLB指令,需要关注它们对TLB查找接口的共用(它们与地址翻译过程共用):

\begin{lstlisting}[language=Verilog, caption={TLB查找接口的共用}]
assign tlb_s1_asid = MEM_tlbsrch_to_csr ? asid_asid_value : 
                     MEM_invtlb_to_csr ? MEM_rj_value[9: 0] : 
                     asid_asid_value;
assign tlb_s1_vppn = MEM_tlbsrch_to_csr ? tlbehi_vppn_value : 
                     MEM_invtlb_to_csr ? MEM_rkd_value[31: 13] : 
                     data_sram_vaddr[31: 13];
\end{lstlisting}
若是TLBSRCH,那么就是用asid和tlbehi的值;若是INVTLB,则要用到相应的rj和rk寄存器的值;否则就是默认的地址翻译使用的asid寄存器以及访存使用的虚地址的值。此外,根据手册,TLBSRCH还要根据查找结果更新TLBIDX寄存器:
\begin{lstlisting}[language=Verilog, caption={TLBSRCH对TLBIDX的更新}]
else if (tlbsrch) begin
        csr_tlbidx_ne <= !tlb_s1_found;
        if (tlb_s1_found) begin
            csr_tlbidx_index <= tlb_s1_index;
        end
end
\end{lstlisting}


对于TLBFILL和TLBWR,只需要拉高TLB的写使能并把相应CSR的值从写端口送入TLB即可。二者的区别在于FILL是随机选一个位置写入,而WR是根据index写入,所以写入位置可以都用TLBIDX的INDEX字段进行指定。
\begin{lstlisting}[language=Verilog, caption={TLB写使能与写index}]
assign tlb_we = tlbwr || tlbfill;
assign tlb_w_index = csr_tlbidx_index;
\end{lstlisting}
根据手册,此处还要注意:
\begin{lstlisting}[language=Verilog, caption={写入TLB的项的e位与g位}]
assign tlb_w_e = !csr_tlbidx_ne || (csr_estat_ecode == `ECODE_TLBR);
assign tlb_w_g = csr_tlbelo0_g && csr_tlbelo1_g;
\end{lstlisting}
\begin{enumerate}
    \item 只有TLBIDX的NE位为0或ESTAT的ecode对应TLBREFILL时(处于重填例外处理过程中),才把写入项的e位置1(有效项),否则写0;
    \item 只有TLBELO0和TLBELO1的g位都为1时,才把写入项的g位置1,否则写0。
\end{enumerate}

对于TLBRD指令,如果查找到的是有效项,则把其页表项信息填入相应csr,并把TLBIDX的NE置0;否则把用于保存页表项信息的csr清零,同时TLBIDX的NE置1。

\subsection{对TLB异常的处理}

\subsubsection{异常的判定与提交}

在流水线内,涉及发送虚地址进行访存的流水级为IF和MEM,因此需要在这两个流水级根据MMU发回的ecode进行TLB异常的判定。这里以MEM为例:

\begin{lstlisting}[language=Verilog, caption={异常检测与传递-MEM}]
always @(posedge clk) begin
    if (rst) begin
        has_exception_out <= 1'b0;
    end
    else if (in_valid && ready_go && out_ready) begin
        has_exception_out <= has_exception || ((|mmu_ecode_d) & (res_from_mem || mem_we));
    end
end
\end{lstlisting}%%%%%%%%%%%%%%%%%%%%%%%%%%%%%%%%%%%%%%%%%%%%%%%%%%%%%%%%%%%%%%%%%%一位and
(|mmu_ecode_d)\ \&\ (res_from_mem || mem_we)表明当前MEM是一条访存指令,并且MMU发回的ecode不为0,说明发生了TLB异常;has_exception则表明当前指令在前面流水级是否发生了异常——将本流水级与前面流水级的情况合并,就得到当前MEM级指令流入RDW后的异常标志has_exception_out。

% 异常码ecode/esubcode的传递仍采用之前实验的思路,若本条指令在前面未发生异常,则把本级的异常码传递下去;否则保持前面的异常码不变(因为发生异常后,指令最后会被冲刷,故在后续流水级其不应该发挥任何功能,所以也不应该再触发异常,所以异常码是第一次“触发”异常时的。例如:若IF级发生ADEF异常,那么取指都变得无意义,更不必提指令执行过程中出现的异常了)。仍以MEM为例:
% \begin{lstlisting}[language=Verilog, caption={异常码的传递-MEM}]
% always @(posedge clk) begin
%     if (rst) begin
%         ecode_out <= 6'b0;
%     end
%     else if (in_valid && ready_go && out_ready) begin
%         if(!has_exception) begin
%             ecode_out <= mmu_ecode_d & {6{(res_from_mem || mem_we)}};
%         end
%         else begin
%             ecode_out <= ecode;
%         end
%     end
% end
% \end{lstlisting}
% 还要将发生异常的地址传递下去,这类似ecode的传递,不再赘述。

异常码ecode/esubcode和异常地址maddr等异常信息的记录、传递与异常的提交仍沿用之前实验的思路,不再赘述。
特别地,IF级存在的ADEF和TLB异常中,ADEF优先级更高,这是因为:进行TLB的查找与TLB异常判定的前提是取指地址合法。
\begin{lstlisting}[language=Verilog, caption={ADEF与TLB异常的优先级区分}]
if(ADEF) begin
    ecode_out <= {6{ADEF}} & 6'h8;
end
else begin
    ecode_out <= mmu_ecode_i;
end
\end{lstlisting}
发生异常后,与前面的实验处理方式相同,需要限制异常指令及其前面流水级指令发挥作用,此处不再浪费多余篇幅描述。

这里还要注意的是,TLB重填异常比较特殊,有单独的异常处理程序入口(ex_tlbr_entry_preserved),在IF级选择取指nextpc时要纳入考虑:
\begin{lstlisting}[language=Verilog, caption={IF级取指PC选择}]
assign nextpc = ex_flush_preserved ? 
                (ex_tlbr_preserved ? ex_tlbr_entry_preserved : ex_entry_preserved) :
                ertn_flush_preserved ? ertn_entry_preserved :
                tlb_flush_preserved ? tlb_flush_entry_preserved :
                csr_flush_preserved ? csr_flush_target_preserved :
                br_taken_preserved ? br_target_preserved : seq_pc;
\end{lstlisting}
(此外,上面这一代码片段中的tlb_flush_entry_preserved和csr_flush_target_preserved分别对应前面提到的因TLB指令/CSR指令而重取指令的起始PC。)

% TLB重填异常的提交也是单独提交:
% \begin{lstlisting}[language=Verilog, caption={TLB重填异常提交}]
% assign ex_tlbr_submit = in_valid && has_exception && (ecode == 6'h3F); // TLB Refill
% \end{lstlisting}


\subsubsection{对CSR的修改}

CSR在应对TLB异常时也需要起辅助作用。比如,需要在TLBEHI的VPPN字段记录发生异常的虚双页号:

\begin{lstlisting}[language=Verilog, caption={TLB异常虚双页号记录}]
if(wb_ex && (  wb_ecode==`ECODE_TLBR || wb_ecode==`ECODE_PIL || wb_ecode==`ECODE_PIS 
            || wb_ecode==`ECODE_PIF || wb_ecode==`ECODE_PME || wb_ecode==`ECODE_PPI)) begin
    csr_tlbehi_vppn <= wb_vaddr[31:13];
end
\end{lstlisting}
\begin{lstlisting}[language=Verilog, caption={TLB重填异常处理入口}]
assign ex_tlbr_entry = csr_tlbrentry_rvalue;  
\end{lstlisting}
发生异常时还需要把出错地址填入BADV寄存器;如果是TLB重填异常,要把TLB重填异常的处理入口地址(TLBRENTRY的内容)发送给IF级;重填异常提交和返回时还要修改地址翻译模式(DA、PG)。这些细节上的处理较为简单,不再赘述。






\section{Debug记录(见下一页)}


\newpage

\subsection{忘记在恰当时机修改地址翻译模式}
本章实验的特点是难度不大,但细节较多,需要仔细研读手册,容易疏忽——尤其是地址翻译模式的设置。

在rst复位时,不能盲目把各种寄存器初始化为0。我们最初的设计中,忘记在rst复位时设置DA为1,PG为0,而是把二者都初始化为0,导致没能进入直接地址翻译模式,影响了取指:
\begin{figure}[H]
    \centering
    \includegraphics[width=0.8\textwidth]{fig/复位地址翻译.png}
    \caption{复位时DA未置为1-波形图}
\end{figure}
正确的设置方式为:
\begin{lstlisting}[language=Verilog, caption={复位时地址翻译模式设置}]
csr_crmd_da <= 1'b1;
csr_crmd_pg <= 1'b0;
\end{lstlisting}


此外,我们最初忘记了在TLB重填异常提交时,要把地址翻译模式改为“直接地址翻译”(DA为1,PG为0);在重填异常处理完后,返回时要改回“映射地址翻译模式”(DA为0,PG为1)。下面展示的片段为修改后的正确逻辑。
\begin{lstlisting}[language=Verilog, caption={TLB重填异常与地址翻译模式修改}]
else if (wb_ex) begin
    csr_crmd_plv <= 2'b0;
    csr_crmd_ie  <= 1'b0;
    if(wb_ecode==`ECODE_TLBR) begin
        csr_crmd_da  <= 1'b1;
        csr_crmd_pg  <= 1'b0;
    end
end
else if (ertn_flush) begin
    csr_crmd_plv <= csr_prmd_pplv;
    csr_crmd_ie  <= csr_prmd_pie;
    if(csr_estat_ecode==`ECODE_TLBR) begin
        csr_crmd_da  <= 1'b0;
        csr_crmd_pg  <= 1'b1;
    end
end
\end{lstlisting}



\subsection{发生TLB异常时忘记限制发送访存请求}

\begin{figure}[H]
    \centering
    \includegraphics[width=0.95\textwidth]{fig/忘记限制req.png}
    \caption{发生TLB异常时忘记限制发送访存请求-波形图}
\end{figure}

上图中,红框部分显示MEM此时是一条store指令,其触发了PME异常(如黄色方框所示,ecode为04),但是访存req信号却仍然拉高(如蓝色方框所示),导致这条发生异常的指令仍然发送了访存请求,显然是不合理的。

经过排查发现,我们忘记在TLB异常发生时限制访存请求req,正确的逻辑为:
\begin{lstlisting}[language=Verilog, caption={在TLB异常时限制访存请求req}]
assign req = in_valid && !handshake_done && !this_flush && (res_from_mem || mem_we) && !this_tlb_flush && !(|mmu_ecode_d);
\end{lstlisting}
即:最后额外“与”上“!(|mmu_ecode_d)”,表示只有没发生TLB异常的访存指令(ecode为0),才能拉高req请求访存。

\newpage

\subsection{忘记使用“流水级有效”信号(in_valid)限制TLB指令功能}

理论上,只有流水级有效时,其内部的指令信息才有意义,才能发挥作用,但我们最初忘记用in_valid信号来限制TLB指令,出现下图波形:
\begin{figure}[H]
    \centering
    \includegraphics[width=0.95\textwidth]{fig/忘记限制tlb指令.png}
    \caption{忘记使用in_valid限制TLB指令功能-波形图}
\end{figure}
红框位置是一条位于MEM级的TLBWR指令,它把TLB的写使能we拉高(见黄框位置),把TLB相关的CSR存放的页表项信息写入TLB(比如图中的tlbehi_vppn),下一拍这条TLBWR从MEM级流出,in_valid拉低表明这时也没有其他指令流入MEM,但是TLB写使能we却仍然拉高(黄框位置)。如蓝框所示,后来出现了一条ID级的CSR写指令,改写了tlbehi的vppn字段;如白框所示,由于TLB的we仍然拉高,导致这一被修改后的tlbehi_vppn被误写入了TLB。

解决方法十分简单,只要把TLB指令“有效”(即会产生作用)的逻辑里“与”上表明流水级有效的in_valid信号即可:

\begin{lstlisting}[language=Verilog, caption={TLB指令“有效”(即会产生作用)逻辑-以TLBWR为例}]
assign tlbwr_to_csr   = in_valid && tlbwr && !this_flush && !this_tlb_flush;
\end{lstlisting}
这一bug具有较强的迷惑性,因为把写好的TLB表项又“写坏了”,导致后续本应提交TLBload页无效异常,却报出了重填异常,表现为异常处理入口点不同,让人首先怀疑是异常入口选择逻辑写错。

\newpage

\subsection{store指令误报出load页无效异常}
如下图所示,黄色方框处的指令报出了红色方框处的pil异常(load页无效异常),而非store页无效异常(pis):

\begin{figure}[H]
    \centering
    \includegraphics[width=0.8\textwidth]{fig/store波形.png}
    \caption{store指令误报出load页无效异常-波形图}
\end{figure}

乍看波形毫无异样,但是查找反汇编文件发现这实际上是st指令,无论如何不可能出现load页无效异常:

\begin{figure}[H]
    \centering
    \includegraphics[width=0.8\textwidth]{fig/store.png}
    \caption{store指令误报出load页无效异常-反汇编}
\end{figure}

回看MEM级生成的用于内存读写选择的wr信号的逻辑——用于生成wr信号的wstrb信号内部包含了与冲刷相关的信号this_flush和this_tlb_flush,即这条指令如果将来会被冲刷,那么一定会强行使得wstrb为0,进而导致wr为0:
\begin{lstlisting}[language=Verilog, caption={MEM级读写选择-wr信号逻辑}]
assign wr = (|wstrb);
assign wstrb  = {4{mem_we && valid && in_valid && !this_flush && !this_tlb_flush}} & (
                    ({4{mem_op[5]}} & (4'b0001 << alu_result[1: 0])) |  // SB
                    ({4{mem_op[6]}} & (4'b0011 << alu_result[1: 0])) |  // SH
                    ({4{mem_op[7]}} & 4'b1111)  // SW;
                );
\end{lstlisting}

而MMU模块内用到wr来区分load/store页无效异常:
\begin{lstlisting}[language=Verilog, caption={load/store页无效异常判定}]
wire pil_d = use_tlb_d && tlb_s1_found && !tlb_s1_v && !data_sram_wr;
wire pis_d = use_tlb_d && tlb_s1_found && !tlb_s1_v && data_sram_wr;
\end{lstlisting}
这两种异常判定的区别仅在于访存操作的类型不同(是读还是写,wr为0还是1)。由于前面在指令“将来会被冲刷”的情况下强行把本应为1的wr置为0,所以会导致store页无效被误判为load页无效。

最终的解决方法是:在wstrb信号中删除表明指令“将来会被冲刷”的this_flush和this_tlb_flush信号,从而使得wr信号不会被强行错误更改,也就解决了上面提到的异常误判问题。

\section{合作说明}

本实验由本组成员共同合作完成,组内同学同等贡献。



\end{document}